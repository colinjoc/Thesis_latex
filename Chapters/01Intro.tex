\chapter{Introduction}
The ability to manipulate our environment and materials to achieve a desired functionality is a human trait that has driven the ever increasing complexity of our society for milenia. Over the past several thousand years the pace of societal and technological innovation has grown rapidly, requiring ever more sophistication to continue this growth. Today, advancement continues in nearly every aspect of our lives at a rate unfathomable to our ancestors. Central to our current technological development is our ability to create the necessary tools, linking technological advancement with the current level of material manipulation we are capable of. That being the case, we are truly living in the nano-materials age, where we have achieved deliberate control and manipulation of materials on the nanometer scale (1 nm = $10^{-9} m$). Nanoscience's reach is vast\cite{bhushan2017}, covering disciplines as diverse as cellular biology\cite{ferrari2005}, catalysis\cite{daniel2004}, energy storage and generation\cite{simon2010,law2005}, and information technology\cite{bez2003,rueckes2000,thompson2006}. Nanomaterials are known to have very different physical properties to their bulk counterparts, the source of which is essentially due to confinement of electrons in the material to a small crystal lattice resulting in a behaviour not seen in large continuous media\cite{datta2005}. The confined electrons give rise to materials properties that are ruled by quantum mechanics, and are referred to as nanomaterials or nanoparticles. Nanomaterials with quantum confinement in at least one direction can be grouped into three classes; pseudo zero-dimensional materials of a small number of atoms referred to as quantum dots and atomic clusters\cite{alivisatos1996,turner2008}, one-dimensional objects that are extended in one dimension and referred to as nanowires or nanotubes\cite{adelung1999,iijima1991}, and two-dimensional objects a few atoms thick such as graphene\cite{novoselov2004} and planar MoS$_2$\cite{mak2010,jadwiszczak2018}. Much of nanoscience and nanotechnologies are concerned with engineering novel nanomaterials with enhanced properties, and this may involve mixing materials of different compositions and dimensionalities. For instance, one can synthesize highly conductive thin-films with superb optical and electrical properties by spreading numerous nanowires randomly over a surface in such a way that they can form a highly interconnected mesh or network. Such material architecture enables the propagation of electrical signals in a wire-by-wire basis, and benefits from the collective aspect of a complex many-body system with emergent properties.

In this thesis, the electrical transport properties of randomly orientated Nanowire Networks (NWNs) is studied with comprehensive computational and theoretical models. These models are succesfully used to explain numerous experimentally observed phenomena and to predict the properties of physical NWNs. Transport in a is examined with two main approaches; first the conductive response of NWNs can be obtained by treating their inter-wire connections as static resistors (this is discussed in chapters three and four of the thesis). In the second case, the network contains dynamic components that change in response to an applied potential and is detailed in chapters five and six.

Due to the random nature of nanowire network connectivity, numerical simulations are necessary to achieve an understanding of the properties of nanowire networks and feature heavily in this thesis. Although numerical analysis provides an excellent lens through which the properties of nanowire networks can be examined, some of the relationships between parameters of a network are best articulated with a mathematical framework. As such, this thesis strives to develop comprehensive theoretical descriptions of nanowire networks where appropriate, and utilise them in conjunction with numerical simulations to illustrate various properties of nanowire networks. 

In this chapter, an introduction to nanowire networks is provided by discussing their fabrication, properties, and applications. A common application of a nanowire network is to use them as a transparent conductor\cite{ye2014,langley2013} and the potential of nanowire networks in this field is discussed in section \ref{Sec: Intro NWN}. In section \ref{Sec: Network Theory}, an introduction to graph and network theory\cite{graph_book} is presented. A discussion on percolation theory\cite{broadbent1957,christensen2002} and its relevance to nanowire networks is given in section \ref{Sec: Percolation Theory}. The potential memory and computing applications of nanowire networks is then introduced in section \ref{Sec: Intro Memristance} through a discussion of the exciting field of resistive switching\cite{kim2011,waser2007,waser2009} and memristive materials\cite{chua1971,strukov2008,memristors2014}, and how these properties have been identified in nanowire networks. Finally, the scope of the thesis is presented in detail in section \ref{Sec: Intro Conclusion}. 
\begin{comment}
\section{Low-Dimension Nanomaterials}
The ability to manipulate our environment and materials to achieve a desired functionality is a human trait that has driven the ever increasing complexity of our society for milenia. Over the past several thousand years the pace of societal and technological innovation has grown exponentially, requiring ever more sophistication to continue this growth. Today advancement continues in nearly every aspect of out lives at a rate unfathomable to our ancestors. Central to our current technological development is our ability to create the necessary tools, linking technological advancement with the current level of material manipulation we are capable of. That being the case we are truly living in the nano-materials age, were we have controlled and deliberate manipulation of materials on the nanometer scale. Nanoscience's reach is vast, covering disciplines as diverse as cellular bioligy to renewable energy.

Nanomaterials are known to have very different physical properties to their bulk counterparts. The source of this difference is essentially due to confinement of electrons in the material to a small crystall lattice resulting in a behaviour not seen in a large continuous medium. Confined nanoparticles can be grouped into three classes, pseudo zero dimensional, one dimensional, and two dimensional objects.  Figure \ref{fig: dos} presents a sketch of the electron density of states for the different dimension nanomaterials and their bulk counterpart. Here the result of axial confinement clearly has an effect on the density of states and the resulting behaviour of electrons in such materials give rise to intruiging physical properties.

The zero dimension particles, or quantum dots, are small a group of hundreds or thousands of atoms\cite{alivisatos1996}. A sketch of the electron density of states of a typical quantum dot is shown in Figure \ref{fig: dos} and the energy states electrons can occupy is represented. Much the same as aquantum the electrons are confined to definite energy levels due to the finite size of the crystal in all three dimensions. A pseudo one-dimensional object has one extended dimension and is confined in the other two, becoming essentially a widthless object. The Carbon Nanotube was reported for the first time in 1991 by Sumio Iijima\cite{iijima1991} and with it heralded a new moment in nanotechnology. The Carbon nanotube combines quantum mechanics dominated nanoscale physics with a macroscopic size, they were found to be incredibly lightweight and have superior strength and stiffness properties (GET CITATIONS). A two dimensional nanoparticle is one that has only one confined dimension. Perhaps the best known example is graphene, a sheet of carbon that is only one atom thick and takes on the honeycomb lattice structure.

\fig{0.75}
{Images/Chapter1/density_of_states.jpg}
{\textbf{Sketch:} }
{sdfdsf.}
{fig: dos} 

In the case of 1-D and 2-D nanoparticles the extended dimensions are essentially infinite in comparison with the confined dimensions. Where the material has a lattice structure the physical properties of the nanopartical can be derived analytically by exploiting the periodicity of the lattice in the extended dimensions and the symmetry in the confined dimensions. For example the electronic density of states of carbon nanotubes and graphene have been calculated with excellent accuracy sining a tight-binding formalism and has been used to show the fascinating properties that can be shown in these materials. 
\end{comment}
\section{Nanowire Networks}
\label{Sec: Intro NWN}
The first report of a metallic nanowire network was made by Adelung et al\cite{adelung1999} in 1999, where the ease, speed and scale of the fabrication technique used to create the network was described. Networks were formed by adsorbing atoms or molecules onto areas of induced strain on a surface to form nanowire networks only bounded by the size of the adsorbing surface\cite{adelung1999}. An image of the network is shown in Figure \ref{fig: nwn_image}(a). Reports of networks comprised of one-dimensional nanomaterials has grown quickly since then, the variety in the comprising material and fabrication techniques expanding alongside this. Among the common materials used in nano-networks are semiconducting nanowires\cite{wu2002}, metallic core-insulating shell nanowires\cite{rathmell2010,chen2013,de2009}, and carbon nanotubes\cite{hu2004,aguirre2006,hecht2006}. A scanning electron microscope (SEM) image of a Ge semiconductor NWN adapted from Wu et al\cite{wu2002} is shown in Figure \ref{fig: nwn_image}(b), and an Ag/PVP core-shell NWN reported in our manuscript\cite{ocallaco2016} is shown in panel (c). The fabrication techniques include, but are not limited to; spray deposition\cite{tenent2009,lu2010,scardaci2011}, drop casting\cite{doherty2009}, spin coating\cite{leem2011}, Mayer rod coating\cite{liu2011}, inkjet printing\cite{finn2015}, and roll-to-roll printing\cite{kim2016}. Each technique has its merits but spray coating in particular leads to a very homogeneous network in terms of wire distribution; it is scalable and can be performed under normal atmospheric conditions \cite{langley2013}. The Ag/PVP NWN in Figure \ref{fig: nwn_image}(c) was fabricated using spray deposition\cite{ocallaco2016}. Networks of different nanomaterials will exhibit unique optimum properties, advantages and limitations. Depending on their design, NWNs can depict enhanced electrical, mechanical, optical, thermal, magnetic, and chemical responses\cite{liangbing2010,langley2014,song2013,wang2015,bobinger2017,sysoev2009} among others, which make them appropriate for specific applications. In this thesis, we shall focus on networks comprised of one-dimensional metallic core nanowires, that are coated with an insulating shell. 

\fig{0.75}
{Images/Chapter1/fig_1.pdf}
{\textbf{Image:} SEM images of nanowire networks reported in the literature.}
{(a) Scanning tunneling microscope (SEM) image of an Rb nanowire network adapted from Adelung et al\cite{adelung1999}. (b) An (SEM) image of Ge semiconducting nanowires adapted from Wu et al\cite{wu2002}. (c) An SEM image of an Ag core PVP shell NWN adapted from O'Callaghan et al\cite{ocallaco2016}.}
{fig: nwn_image}

A particularly exciting property of nanowire networks is their high transparency and high attainable conductivity\cite{langley2014,song2013}. Thin films that couple high electrical conductivity and optical transparency are crucial in a number of applications including flexible electronic displays and touch-screens. Currently this market is dominated by transparent conducting oxides, in particular Indium and Florine doped Tin oxide\cite{kumar2010}. There are three major drawbacks with incorporating Indium Tin Oxide into transparent conductors. Firstly Indium itself is relatively scarce. Though it has a similar abundance in the Earth's crust as Silver, roughly 50 parts per billion, it has few naturally occurring minerals and is mainly produced as a by-product of Zinc refinement\cite{phipps2008}. The second issue with Indium Tin Oxide is brittleness\cite{chen2001}, a limiting factor in its inclusion in flexible transparent conductors, a market that is expected to grow quickly in the coming years. Finally, the deposition of Indium Tin Oxide onto a substrate is performed using sputtering techniques\cite{meng1998}, where material is deposited onto a target substrate as a vapour. This requires high temperatures and results in slow deposition rates\cite{meng1998}. Furthermore, a large deal of the vaporised Indium Tin Oxide does not deposit on the target substrate, requiring recapture methods that further add to the expense and time in transparent conductor production\cite{stewart2014}. Nanowire networks are not limited by these issues and have potential to be the dominant component in future transparent conductors.

Nanowire networks have shown optical transparency and sheet resistances comparable with Indium Tin Oxide\cite{lagrange2015,hsu2013,liangbing2010}, which demonstrates nanowire networks potential in optoelectrical devices. Fabrication techniques of nanowire networks are scalable and inexpensive, and nanowire networks have been shown to be very flexible while maintaining their high transparency and conductivity. It is unsurprising that their potential as transparent conductors has largely driven their development over the past two decades\cite{bellet2017,gong2017,langley2013}. This development has lead to nanowire networks being incorporated into LED displays\cite{zeng2010,yu2011}, and thin-film solar cells\cite{morgenstern2011,lan2013,song2013,dechan2015} where they have been shown to perform similar to Indium Tin Oxide devices. It has been shown by Madaria et al\cite{madaria2010} that Ag NWNs can remain conductive when bent up to $160^{\circ}$ and returned to their original sheet resistance when the bending stress was removed. Lim et al \cite{lim2012} examined various mechanical properties of Ag NWNs. NWNs were bent, twisted and put under torsional stress with little change in sheet resistance. This flexibility of nanowire networks makes them ideal candidates for the development of flexible transparent conductors which could be used to develop flexible displays or incorporated into wearable devices\cite{langley2013,gong2017}. 

Besides their applications as transparent conductors, nanowire networks have properties that are suitable for various other applications. Many examples of nanowire network-based transparent heaters, crucial for anti-fog windows, are found in the literature\cite{celle2012,sorel2014,ergun2016,bobinger2017}. The scalable fabrication processes such as spray deposition enable large scale surface coatings of nanowires meaning that large curved surfaces can be easily coated in nanowires to form a transparent heater. Nanowire networks have also been successfully applied to non-optical devices such as sensors\cite{takei2010,wang2015}, fuel cells\cite{chang2016,chang2014,chang2013}, and thin-film thermo-acoustic speakers\cite{la2017}. The many applications of one-dimensional nanomaterial networks has necessitated theoretical descriptions of their properties in order to tune their characteristics. In this thesis, we shall focus on the resistive properties of a variety of NWN materials, which is achieved by mapping NWNs into circuit grid models comprised of lumped circuit elements. For instance, to extract the resistances of a NWN, the system is modeled as a network which responds to voltage/current sources. In the next section, an overview of network theory is given to illustrate one of the main modeling approaches used in this thesis.

\section{Network Theory}
\label{Sec: Network Theory}
Network theory has its roots in the early 18$^{th}$ century where Leonhard Euler played a key role in its early development. A well known problem of the day was the "Seven Bridges of K\"onigsberg", where it was questioned if a route was possible that crossed the seven bridges in the city exactly once. Euler solved the problem using a method that sowed the seeds to what would become graph theory\cite{guichard2017}, of which network theory is a subdiscipline\cite{pozrikidis}. 

A network is a collection of nodes that are connected in some way by edges. The network may represent some physical entity such as bridges and routes between them as in the seven bridges of K\"onigsberg problem, or a more abstract construction such as people and their inter-personal relationships in a social network or agents and transactions in an economic network\cite{pozrikidis}. Network theory is concerned with the study of these representaions of relations between objects\cite{pozrikidis}. Among the many applications of network theory, we shall focus on transport of an entity or a signal through a network. In this thesis, the transported entities are current and charge; the associated mathematics are given in chapter 2 and can be applied to other transport problems such as the propagation of heat or mass. 

A many-body system can be abstracted into a network form where nodes are individual particles and edges are the inter-body interactions. These class of problems can be simplified massively by making use of underlying system symmetries. For example, consider a tight-binding model applied to an infinite periodic lattice of atoms, the electron density of states of such a system is solved using Fourier transforms along the directions of symmetry\cite{datta2005}. The same thinking can be applied to resistive lattices and in chapter 2, Cserti's method for calculating the inter-node resistance in an infinite resistive lattice is presented\cite{cserti2000}. 

Network theory can be applied to NWNs to calculate their electrical properties with an appropriate mapping of the NWN onto a mathematical network. In chapter 3, mappings between a nanowire network and a mathematical network are introduced and are then solved using the network transport mathematics layed out in chapter 2. Due to the spatial randomness associated with disordered nanowire networks, there are no symmetries to exploit in order to create an analytic function capable of solving the resistive properties of a network, like the method outlined by Cserti\cite{cserti2000}. As shown in the previous section, many applications of nanowire networks require specific sheet resistances and optical transmission values that depend on the length and diameter of the wires, device size, and nanowire densities. Thus it is necessary to quantitatively understand how the sheet resistance depends on these physical features. Due to the spatially disordered nature of NWNs, altering these parameters will change the connectivity, and consequently the resistance, of a network\cite{ocallaco2016}. To remove a degree of disorder from analysis, an ensemble of networks are studied to identify how properties such as the resistance depends on connectivity altering parameters\cite{ocallaco2016,pike1974}. Another approach widely used to study the conduction properties of random NWNs is percolation theory\cite{pike1974}, which can be used to relate NWN resistance with numerous network characteristics, and a description of this is presented in the following section. 

\section{Overview of Percolation Theory}
\label{Sec: Percolation Theory}
Percolation theory is concerned with the behaviour of connected clusters in a network. In 1957, Broadbent and Hammersley\cite{broadbent1957} introduced the concept of modeling how a fluid percolates through a porous medium, drawing analogy with electrons flowing through a lattice or disease through a population. As opposed to a diffusive process, where the particles themselves are a source of stochasticity as they move through a medium, they defined a percolative process as where the medium is the source of stochasticity and completely determine the movement of particles through them\cite{broadbent1957}. To model the porous material as a percolative process, consider a square lattice of size $n\times n $ nodes, each node has four nearest neighbours (order number $z=4$). The nearest neighbour edges were connected with some probability $p$, or no connection existed with a probability $1-p$. There exists some critical bond probability $p_c$ at which there exists a connecting path between the two extreme ends of a network, below $p_c$ no path exists. Figure \ref{fig:percolation_sq} is a visualisation of a percolative square lattice where $p>p_c$ in panel (a) and $p<p_c$ in (b). The thick blue lines on either side of the two networks represent the two boundaries between which percolation may occur and the blue lines between nodes are connecting edges.
\fig{1}
{Images/Chapter1/percolation_sq.pdf}
{\textbf{Sketch:} A visualisation of percolation in a square lattice above and below the critical bond value}
{(a) A sure lattice with nodes depicted as red circles and connecting edges as blue lines. The edges between nearest neighbours exist with a probability $p = 0.6$ which is greater than the critical probability $p_c = 0.5$ for a square lattice\cite{pike1974}. (b) A square lattice where the edges between nearest neighbours exist with a probability $p = 0.3$ which is less than the critical value and so a percolating path across the network does not exist.}
{fig:percolation_sq}

In a 1974 paper, Pike and Seager\cite{pike1974} extended the concept of percolation theory to random networks formed by different objects that were randomly distributed and connected over a defined two-dimensional area\cite{pike1974}. This is a departure from previous works where the positioning of nodes and edges were pre-determined in a grid template and their existence was given by some probability distribution. Among the many types of objects studied, the percolative characteristics of one dimensional objects (sticks) were examined\cite{pike1974}. Here the requirement for a connection to form between two sticks requires their centers to lie within a distance $L$ of one another, $L$ being the length of the sticks, and that their relative orientations are such that they intersect. Pike and Seagar used percolation theory to show that the critical density of two-dimensional randomly oriented sticks ($(n_w)_c$) can be calculated using the expression
\begin{equation}
(n_w)_c L^2 = Q
\label{eq: critical_density}
\end{equation}
where Q is a constant. The stick percolation model can be used to estimate the critical wire density of a nanowire network by considering its wires as ideal one-dimensional sticks. 

Percolation theory has been used to explain the resistive properties of a network of conducting wires\cite{balberg1983}. A transition from a non-percolating to a percolating network occurs at a critical bond probability $p_c$, and is a simple example of a continuous phase transition\cite{christensen2002}. The scaling between sheet conductance $\Gamma_s$ and the wire density ($n_w$) can be described with a power law typical in continuous phase transition\cite{balberg1983}
\begin{equation}
\Gamma_s \propto (n_w - (n_w)_c)^{\beta}
\end{equation}
where $(n_w)_c$ is a critical wire density below which a percolative path does not form and so the system is not conductive. This scaling only holds for wire densities near to the critical value $(n_w)_c$. For networks where all wire lengths are identical and wire densities are in the criticality region, where $n_w \gtrsim (n_w)_c$, Li and Zhang have shown $\beta = 1.280 \pm 0.014$. For wire densities beyond the criticality region, Li and Zhang have shown that the conductivity exponent $\beta$ depends on both the junction resistance $R_j$ and the intra-wire resistance $R_{in} = \tilde{\rho} L$ where $L$ is the length of each wire in the system, and $\tilde{\rho}$ is the resistivity per unit cross sectional area\cite{li2009}. \v{Z}e\v{z}elj and Stankovi\'c \cite{zezelj2012} have shown that exponents have a dependence on wire density as well as the ratio of $R_j/R_{in}$ and can vary between $1 < \beta <2$ for sigificantly large wire densities. 

Percolation theory provides an excellent understanding of the properties of NWNs near critical wire densities and has been used in our manuscripts and in many other works\cite{ocallaco2016,fairfield2016,de2011}. In particular, the critical wire density provides an estimate for the minimum wire density in networks whose constituent wires are all of identical length\cite{fairfield2016,ocallaco2016}. The power law trend of conductivity with wire density provides a useful qualitative comparison with experimental and simulation results, however, as previously mentioned, the exponent of which depends on many factors in a non-transparent way. For many applications, a quantitative expression for the conductivity of a network that is a function of the relevant nanowire properties is necessary. This brings us to one of the goals of this thesis, that is to develop an approximation for the sheet resistance of a nanowire network in terms of all of the nanowire properties. This approximation that takes into account the physical properties of a nanowire network and does not require empirical fitting is presented in chapter 4. 

The discussion so far regarding resistive networks, in particular nanowire networks, has involved static materials that behave as Ohmic resistors. In the following section, materials that have an adapative response to electrical stimulus and how they are related to nanowire networks are discussed. 

\section{Memristive Behaviour of Nanowire Networks}
\label{Sec: Intro Memristance}
Until now the discussion of NWNs has considered static networks where the resistive elements are unchanged by current-flow. Recently it was shown that under certain circumstances, NWNs have a memristive response to current-flow\cite{scaling2018}, that is their resistance change according to the amount of current-flow through the network. Here is an overview of memristive systems and how they pertain to NWNs.

\subsection{Memristor and Memristive Systems}
In 1971, Leon Chua introduced the concept of a memristor, a memory-resistor, by characterising the relationship between the charge $q(t) = \int_{-\infty}^{t} I(\tau) ~d\tau$ and the magnetic flux-linkage $\phi(t) = \int_{-\infty}^{t} V(\tau) ~d\tau$, where $I(\tau)$ and $V(\tau)$ are functions describing the historical applied current and voltage respectively\cite{chua1971}. Chua argued that by symmetry, there ought to be a fourth (nonlinear) fundamental circuit element besides the resistor, capacitor and inductor. The memristor has an associated memristance ($M$) which is related to the voltage and magnetic flux-linkage as
\begin{align}
& M = \frac{d\phi}{dq} \label{eq: mem_phi}\\
& M(q(t)) = \frac{d\phi/dt}{dq/dt} = \frac{V(t)}{I(t)} \label{eq: memristance_ohm}
\end{align}
Here, by expanding on the definition of a memristor, one finds that it will take the form of a resistance in Ohm's law, but since $q(t)$ and $\phi(t)$ are time-dependent integrals, $M$ is not constant and is in fact a tunable resistance depending on the history of applied current and voltage. For example\cite{chua2011}, consider the relationship between charge and flux in a two terminal memristive device and a sinusoidal applied current with the following relationship between charge and flux linkage,
\begin{align}
\phi(q) &= q + \frac{q^3}{3} \label{eq: link_flux} \\
I(t) &= A ~ \sin(\omega t) \label{eq: current_profile}
\end{align}
where $q$ is the charge, $\omega$ the frequency, and $A$ the amplitude of the input current.
Performing the time integral for the cumulative charge, we obtain
\begin{equation}
q(t) = \int_{-\infty}^{t} A \sin(\omega \tau) ~d\tau = \frac{A}{\omega} ( 1 - \cos(\omega t))
\label{eq: charge}
\end{equation}
The linkage-flux is then obtained by substituting the result of equation \ref{eq: charge} into equation \ref{eq: link_flux}, i.e.
\begin{equation}
\phi(t) = \frac{A}{\omega}(1-\cos(\omega t) ( 1 + \frac{A^2}{3 \omega^2} (1 - \cos(\omega t))^2 
\end{equation}

The voltage across the system is the time derivative of the flux
\begin{equation}
V = \frac{d \phi}{dt} = M(q(t)) I(t)
\end{equation}
After performing the time derivative the memristance of the circuit element can be isolated as $M(q) = 1 + q^2$ or 
\begin{equation}
M(q(t)) = 1 + \left( \frac{A}{\omega}( 1- \cos(\omega t) \right)
\label{eq: memristance_charge}
\end{equation}
An I-V curve of a memristive system from this example with $\omega = A = 1$ can be seen in Figure \ref{fig: mrm_example}(a); the non-constant memristance is captured by the pinched hysteresis curve. The relationship $M(q) = 1 + q^2$ which is plot in Figure \ref{fig: mrm_example}(b) shows that memristance is finite for a finite charge or current-flow. Taken in conjunction with equation \ref{eq: memristance_ohm}, the voltage over a memristor is zero for when there is no current-flow, giving the pinched hysteresis I-V curve seen in Figure \ref{fig: mrm_example}(a). This relationship between the memristance and charge also shows that the memristance can be tuned to any level\cite{chua2011}. Since charge modulates the memristive response, the charge is referred to as a state variable. By simply sending current pulses through the memristor its value can change accordingly. Here the memristance of an element can be captured by coupling the response and the state equation as
\begin{align}
&V = M(w) I\\
&\frac{dw}{dt} = I
\end{align}
where the state variable $w$ is the charge in the example shown above. An ideal memristor is one that depends soley on the charge ($q$).

\fig{1}
{Images/Chapter1/memristance_example.pdf}
{\textbf{Plot:} The memristance of a theoretical circuit element demonstrated by a hysteresis I-V curve with a tunable memristance, and a memristive response of a experimental sample. }
{Top panels adapted from Chua\cite{chua2011}. (a) An I-V curve for the memristor determined by equations \ref{eq: link_flux} and \ref{eq: current_profile} with $A = \omega = 1$ that displays the pinched hysteresis curve characteristic of memristors. (b) Visualisation of relationship \ref{eq: memristance_charge} (R being the memristance) and shows the tunability of the memristance. Bottom panels adapted from Strukov et al\cite{strukov2008}. (c) The top panel is the normalised doped-layer length ($w/D$) is plot (red) and the voltage reponse (blue) for a simulated memristor that follows the ion drift model in equation \ref{eq: ion-drift-intro} by Strukov et al\cite{strukov2008}. The bottom panel is the associated I-V curve. (d) I-V curve of a memristor measured by Stewart et al\cite{stewart2004} that Strukov et al compared with the ion-drift model\cite{strukov2008}. A sketch of the Pt-TiO$_2$-Pt device is visualised in the top left corner.}
{fig: mrm_example}

Strukov et al\cite{strukov2008} first reported a phsycial system that had memristive properties along with an ion-drift model to capure the dynamics of the device. In their model, Strukov et al hypothesised that the memristance was modulated by an interfacial boundary between an undoped $TiO_2$ and a $TiO_{2-x}$ layer doped with oxygen vacancies. The electrical response of such a junction can be modelled as
\begin{align}
V(t) & = \left[ R_{on} \frac{w(t)}{D} + R_{off} \left( 1 - \frac{w(t)}{D} \right) \right] I(t) \nonumber \\
\frac{dw}{dt} & = \mu_v\frac{R_{on}}{D} I(t) 
\label{eq: ion-drift-intro}
\end{align}
where t is time, D is the full length of the $TiO_2/TiO_{2-x}$ junction, $\mu_v$ is the mobility of the ions, $I$ is the current, $V$ is the output potential of the device, and $R_{on}/R_{off}$ are the low/high resistance states. The state variable $w$ is the length of the doped layer which modulates the resistance of the junction and can vary between $0$ and $D$. The resistance of the junction clearly varies between $R_{on}$ and $R_{off}$ depending on the value of $w$. Figure \ref{fig: mrm_example}(c) presents the normalised filament width $w/D$ (red) and the voltage response (blue) for a sinusoidal current source in the top panel, and an I-V curve of a device corresponding to the ion-drift model outlined in equation \ref{eq: ion-drift-intro} in the bottom panel. Figure \ref{fig: mrm_example}(d) is an experimental I-V curve of a Pt-TiO$_{2-x}$-Pt device reported by Stewart et al\cite{stewart2004} that Strukov et al succesfully compared with their ion-drift model, and so classified as a memristor\cite{strukov2008}. 

Chua and Kang generalised the concept of a memristor to have a number of state variables that need not solely be the charge flowing through the system. The memristance and state equations can be generalised as
\begin{align}
V & = M(w,I)I\\
\frac{dw}{dt} & = f(w,I) 
\end{align}  
where w is a set of state variables and the functions $M()$ and $f()$ can be explicit functions in time\cite{chua1976}. For example in Strukov et al's model, the state variable is the width of the doped $TiO_2$ layer. With this generalised definition of memristance, a host of materials undergoing a phenomena known as resistive switching\cite{pan2014,yang2013,valov2013} were classified as memristive systems\cite{chua2011}. 

\subsection{Resistive Switching Phenomena}
The concept of a non-constant resistive device is not an entirely novel idea in physics. Resistive switching devices, capable of cycling between a High Resistance State (HRS) and a Low Resistance State (LRS) is a rich and active field of research. As shown in Figures \ref{fig: mrm_example}, Strukov et al first identified that a resistive switching material could be related to a memristor\cite{strukov2008}, thus linking the two fields. The force driving much of the development of resistive switching research is their potential for memory devices, which have come to be labeled as Resistive Random Access Memory (RRAM) devices\cite{pan2014}. RRAM devices have been shown to have excellent physical properties for memory applications\cite{yang2013,valov2013} such as; high area compaction\cite{lim2015,wang2015}, high state switching speeds (<100 ps)\cite{choi2013}, good state retention times (100's years)\cite{chien2011,wang2012}, high switching endurance (>$10^{12}$ cycles)\cite{lee2011}, and low power consumption\cite{yang2013,valov2013,waser2007}. Perhaps the most important listed property of the resistive switching, or memristive devices is their scalability. Traditional silicon transistor technology is fast reaching the natural barriers that quantum mechanics pose\cite{thompson2006}, thus threatening the exponential growth in memory storage the industry has strived for over the past decades, and RRAM devices could overcome these spatial limitations. 

A common architecture for a memristor is two metallic layers separated by an insulating barrier, referred to as a metal-insulator-metal (MIM) device. The mechanism that regulates the memristive response depends on the material characteristics of the device. Examples of memristive MIM materials\cite{yang2013} are transition metal oxides\cite{strukov2008,gale2014}, amorphous-to-crystal phase materials such as GeSbTe\cite{wuttig2007}, and polymeric matrices sandwiched by metals (e.g. Ag/PVP plates)\cite{yang2012,scaling2018}. During the breakdown of a MIM junction, the growth of a conducting filament bridging the metal plates takes place and this can be regulated by distinct mechanisms including thermochemical (TCM), electrochemical metallisation (ECM), and valence change (VCM) \cite{memristors2014,lim2015,jeong2012,manning2017,waser2009}. With the gradual filament growth, a drastic reduction in the characteristic resistance of the junction can be measured. By controlling the current-flow through the device, the conductive filament can be forced to expand or contract making the conductive filament reversible\cite{yang2012}. This reversibility allows devices to be controllably switched between high and low resistive states\cite{yang2012}.

\fig{1}
{Images/Chapter1/unipolar-bipolar.png}
{\textbf{Sketch:} Examples of bipolar and unipolar resistive switching.}
{(a) Sketches of a unipolar switching and (b) a bipolar switching between a HRS (red) and a LRS (purple) in an I-V sweep. $I_{cc}$ is the current compliance. Adapted from Lim and Ismail\cite{lim2015}.}
{fig: uni bi polar}

Before discussing the mechanisms that regulate resistive switching devices in more detail, let us discuss two types of resistive switching dynamics typical in memristive junctions which are presented in Figure 1.4. Panel (a) is referred to as unipolar and panel (b) as bipolar resistive switching\cite{ielmini2015}. Figure \ref{fig: uni bi polar} presents sketches of I-V sweeps for ideal unipolar and bipolar switching devices. Unipolar switching occurs when the transitions between the HRS and the LRS occur with the same polarity, whereas in bipolar switching, the opposite polarity is required to rupture the filament. Also shown in Figure \ref{fig: uni bi polar} is the so-called current compliance, $I_{cc}$, that is set in experiments as an upper limit to current-flow in I-V sweeps. This is to protect the device from large current-flows which can cause an irreversible change in resistance state or damage the device. 

In electromechanical metallisation (ECM) memristors\cite{waser2009}, the conductive filament is built between the metallic layers by means of cation transport. Various other names have been associated with this form of memristor, mainly conductive bridge random access memory\cite{jameson2013} (CBRAM) and programmable metallisation cell\cite{russo2009} (PMC). Here a highly electromobile metal electrode such as Cu or Ag, known as the active electrode, acts as the source of material for the conductive filament that nucleates on the opposite electrode and grows back towards the cathode\cite{menzel2017}. Figure \ref{fig: filament_growth} presents a sketch of an Ag/Pt ECM cell at various stages of conductive filament growth and the associated I-V sweep for the device\cite{waser2009}. In panel (a), the electric field is sufficient to cause Ag cations to begin to migrate through the insulating layer to the counter electrode and grow a conductive filament back towards the active electrode. This is referred to as the SET procedure. The device is limited to a current compliance shown as the dashed horizontal line and a conductive filament that has bridged the inter-electrode insulator is shown in panel (b). The memristor is now in the ON state. The current is driven at the opposite polarity and the conductive filament ruptures with some of the Ag atoms returning to the active electrode in the RESET procedure. A visualisation of this is in panel (c). The pristine memristor is shown in panel (d) where there is no filament formation and it is labelled as the OFF state. The observation of a conductive filament growth was reported by Yang et al\cite{yang2012} where Ag filaments were observed in an Ag active electrode and Pt inert electrode system and are shown in Figure \ref{fig: exp_filament}. In panel (a), the device is imaged after a forming process. In the insulating gap between electrodes, several distinct filaments are observed, most notably the top-most filament which appears to span the entire insulating layer. A zoomed image is presented in the red square of this particular filament at the inert electrode interface. In panel (b), the filament was ruptured electrically which is evidenced in the zoomed area of the filament near the electrode which is no longer connected. 

\fig{0.75}
{Images/Chapter1/new_filament_growth.png}
{\textbf{Sketch:} Sketches of filament growth in an Ag ECM memristor cell with a corresponding I-V curve.}
{An I-V sweep measurement taken on a bipolar Ag/Pt memristor ECM cell with a Ge$_{0.3}$Se$_{0.7}$ inter-electrode insulator. The panels (a)-(d) are sketches of the state of the conductive filament at various stages in the I-V curve. Adapted from Waser\cite{waser2009}.}
{fig: filament_growth}

In valence change mechanism (VCM) devices, the memristance is mediated by field-assisted migration of oxygen anions in transition metal oxides and the resulting valence change of the cation sublattice\cite{waser2009}. The active materials are transition metal oxides, common examples are $HfO_x$, $SrTiO_3$, $ZNO$, $AlO_2$, and $TiO_2$\cite{gale2014,waser2007}. In fact it was a $Ti0_2$ based device that was first linked to memristance by Strukov et al\cite{strukov2008}. The thermochemical mechanism (TCM) for memristance is based on stoichiometry change and redox reactions in the oxide layer due to current induced heating\cite{ielmini2011}. As it is predominantly Ag nanowire networks that shall be discussed in this thesis we shall focus on ECM conductive filament formation as the mediator of a memristive response in a NWN\cite{scaling2018}.

\fig{0.75}
{Images/Chapter1/exp_filament_label.png}
{\textbf{Sketch:} Experimental images of filament growth in an Ag/Pt ECM memristor cell.}
{Experimental observations of conductive filament growth in an Ag/Pt memristive ECM cell. The thin structures highlighted by arrows in (a) are the Ag filaments growning from the Pt electrode towards the Ag electrode. A zoomed image of the top filament is shown in the red square showing a connection with the Pt electrode. (b) The filament after a RESET operation was performed on the device. The filaments have shrunk, the longest one has ruptured its connection with the electrodes. The zoomed area shows no connection with the Pt electrode. The scale bars represent 200 nm. Adapted from Yang et al\cite{yang2012}.}
{fig: exp_filament}

\subsection{Potential for Neuromorphic Computing}
Asides from RRAM applications, memristors have potential as central components in other novel devices such as multi-bit memory storage\cite{valov2013,russo2009} and neuromorphic (brain-like) computation devices\cite{ziegler2012,riggert2014,calimera2013,ye2014mem,gkoupidenis2017,kumar2017,yu2013,indiveri2013} due to their tunable resistance levels. The brain is a highly complex machine formed by billions of neurons which are disorderly interconnected by trillions of synapses\cite{jo2010,yu2013}. Our brain has unique abilities that outperform by far the fastest computers on the planet such as ultra-fast sensory processing, high-level pattern recognition, and the ultimate skill of learning from experience. Brain activity is also incredibly energy-efficient; it consumes about 20 W, equivalent to a dim light bulb\cite{sengupta2014}. To date there has been numerous attempts to mimic biological computation through simulation on traditional von Neumann computer architectures \cite{indiveri2013}. However this approach is computationally expensive and thus energy intensive. Another approach to achieve biological computation is through the use of neuromorphic computing architectures\cite{zahari2015,prezioso2015,yang2013}. These can be decentralized networks of memristors that emulate the behaviour of biological neurons and synapses\cite{jeong2013}. While these architectures are much more energy efficient, the fabrication of such devices can be quite difficult, often requiring exact engineering of individual memristor components and connections. For instance, a common method of realising networks of memristors is so-called crossbar arrays\cite{hu2012,prezioso2015,xia2009,joshua2012,borghetti2009}, in which memristive junctions are arranged in an ordered square grid by patterning nanowires transversely and longitudinally over the device area. A high level of component homogeneity and regularity in neuromorphic networks may not be required as the variability, stochasticity and component reliability which are becoming increasingly difficult to overcome in traditional computing technologies do not pose as big a problem to biological computing systems\cite{querlioz2013}. Indeed the variability of individual synapses and the complexity of the global synapse network are exploited to perform robust and reliable computations, all while using a fraction of the power that a von Neumann computer would need for similar performance. Such attributes have inspired the creation of the so-called neuromorphic devices that have the potential to revolutionize computing technology with the next-generation of microprocessors that will mimic brain functions\cite{calimera2013,ye2014mem,gkoupidenis2017,kumar2017,yu2013}.

A nanowire network is essentially a collection of highly connected metal-insulator-metal junctions that can exhibit (or not) memristive features. As stated previously, nanowires are coated in a insulating material to prevent flocculation in solution and to aid in synthesis\cite{shah2004}. NWNs can be annealed to remove the insulating barrier separating the metallic nanowire cores in order to maximise the optoelectrical properties of the network, leaving it at a non-varying high conductance state\cite{bellew2015,rocha2015}. This is suitable for transparent conductor applications, however annealing does not favour neuromorphic applications that require the adaptive properties of a MIM cell. Nanowires with a memristive response to current-flow can lead to interesting collective behaviours in a NWN that resemble those in biological neural networks. A memristive NWN has much potential for memory devices as discussed above and in brain-like computing, and their properties are discussed in more detail in chapters 5 and 6. In the next section the goals of this thesis are presented and the layout of the thesis is presented.
% In fact, the dynamic memristive response of NWNs to current has been previously reported; Nirmalraj et al reported the tunability of Ag NWNs conductivity with current flow\cite{nirmalraj2012} but the link with memristive devices was not made at the time.
%=====================+
\section{Thesis Goals and Scope}
\label{Sec: Intro Conclusion}
A brief overview of several fields related to nanowire networks was presented in this chapter and provides context for the work that will be discussed in subsequent chapters. As seen in section \ref{Sec: Intro NWN}, NWNs have great potential as components in transparent conductors, an application which is sensitive to the transparency and sheet resistance of the network. These properties are mediated by the fundamental resistive and geometric aspects of the network which have been highlighted by numerous works in the literature, many of those conducted by co-workers at Trinity College Dublin\cite{de2011,lyons2011,de2009,bellew2015,lyons2008}. Percolation theory had been succesfully applied in these works to understand the connectivity and resistive response of such networks, however there has always been a lack of a more quantitative computational toolbox whose outcomes could be directly related to measurable quantities. In other words, rather than employing qualitative theoretical views to explain trends and asymptotic behaviours in a given NWN experiment, this work is dedicated to the development of a wide range of advanced computational and theoretical models that can acurately describe the properties of highly disordered nanomaterials such as the NWN. 

In this way, one of the first objectives of this thesis was to expand the understanding network resistance and connectivity, and how to link it to the complex connectivity profile these highly-disordered materials can display \cite{rocha2015,fairfield2016}. Another goal of this thesis is to derive analytical expressions for various geometrical and resistive quantities in a NWN which can be used to calculate sheet resistances with a closed-form expression\cite{ocallaco2016}. These topics are discussed in chapters 3 and 4 in this thesis.

In recent publications by prominent experimental groups at Trinity College Dublin and co-workers, the dynamic response of NWNs to electrical stimulus was reported\cite{nirmalraj2012,fairfield2014,bellew2014,rocha2015,fairfield2016,manning2017}. Individual electrically contacted nanowires were shown to have a memristive reponse to optical and electrical stimuli\cite{okelly2016}, and such emergent responses were also found in nanowire network samples\cite{nirmalraj2012,bellew2014,manning2017}. Several curious relationships between the memrsitance of individual junctions and that of networks were highlighted, and required computational simulations to fully understand them. Another challenging goal of this thesis is to develop a computational model to capture the memristive nature of nanowire junctions and to simulate highly connected networks of such junctions\cite{scaling2018}. The dynamic response of NWNs at very low current levels had been previously captured using a leaky capacitor model, in which junctions can respond as a capacitor to charge accumulation and can undergo breakdown at some critical voltage drop value\cite{nirmalraj2012,fairfield2014,fairfield2016}. I applied both memristive and capacitive pictures in real-world problems which consisted of explaining and reproducing experimental data gathered by my co-workers in the Nanoscale characterisation and processing group of Trinity College Dublin\cite{scaling2018}. The memristive and capacitive dynamic responses of NWNs are addressed in chapters 5 and 6 of this thesis. Below is a more comprehensive outline of the topics covered in each of the following chapters.

In chapter 2, the background theory and mathematical methods used throughout this thesis are introduced. The rest of the thesis can be broken into two sections, the first one deals with annealed networks made of highly conductive junctions, and is motivated by transparent conductor applications. Chapter 3 presents two mappings between a nanowire network onto a graphical representation such that the electrical properties of a network can be calculated using network theory. One of the mappings only considers inter-wire junctions as a source of resistance in a network, while the other also considers the resistance of nanowires themselves and so its impact on network conductance can be examined. A novel method to digitise nanowire network geometry is also introduced, reducing the spatial uncertainty from comparisons between experimental measurements and computer simulations. These comparisons were used to approximate the junction resistance of Ag/PVP nanowires. In chapter 4, an original approximation for the sheet resistance of a nanowire network in terms of many of the nanowire properties is presented. To achieve this, analytical expressions are derived to determine the number of resistive elements in a network which is used to calculate an effective medium that describes the resistance of a NWN. The approximation is successfully used to estimate the ultimate conductivity attainable if the network junctions are annealed to negligible resistance values. The effectve medium technique is of particular note as it requires no data-fitting to achieve accurate results.

The second part to this thesis deals with unannealed nanowire networks whose inter-wire junctions are extremely resistive in low current regimes but can respond dynamically to current-flow. In chapter 5, the junctions are shown to behave as memristive elements in response to regulating currents. The properties of a network of such memristive junctions is shown to behave similarly to the nanowire, representing an emergent self-similarity between the network and the junctions. For certain nanowire properties, highly localised current-flows emerge in the network, showing a "winner-takes-all" behaviour where one path dominates network conductance. A multi-electrode device is simulated operating within the memristive picture and highlights some important properties of winner-takes-all paths that could be exploited for neuromorphic and memory applications. The memristive response is compared with a capacitive junction model that applies to negligible current ranges in chapter 6. The two models are shown to capture distinct dynamics in the electrical activation of junctions with the capacitive model displaying scale-free complex dynamics and both having different levels of fault tolerance. Chapter 7 contains the thesis conclusion as well as future research that follows the results of this work. 
