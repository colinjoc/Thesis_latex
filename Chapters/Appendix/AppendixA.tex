\chapter{Multi-Nodal Electrodes in a Square Lattice}

In this section the Inter-nodal Resistance Green's Function is generalised to a system with extended electrodes and where lattice is infinite in the $\vec{a}_1$ direction but finite in the $\vec{a}_2$ direction. The width of the system in this direction is $N_y$ nodes. The injection electrode is represented by a strip of nodes at positions $\vec{r}_{in} = x_{in} \vec{a}_1 + m \vec{a}_2, ~ 0\leq m < N_y$. Similarly the extraction electrode is represented by a strip of nodes at positions $\vec{r}_{out} = x_{out} \vec{a}_1 + n \vec{a}_2, ~ 0\leq n < N_y$. In order to simplify notation, let $x_{in} = 0$ and $x_{out} = N_x$. The number of nodes in each electrode (or the width of the electrode) is $N_y$ nodes, the same as the width of the system. See Fig.\ref{fig:square_elec} for a schematic of the system.

Following the derivation for the Inter-nodal Green's Function, one begins by describing the current at lattice point $\vec{r}$. A total current $I$ is injected at one of the electrode with a current $i_0 = \frac{I}{N_y}$ injected at each of the nodes in this injection electrode. 
Consider a current $i_0$ is injected at position $(0,m)$ and $i_0$ is extracted from the  entire extraction electrode. We define the current extracted at each individual node $(N_x,n)$ as $i_{mn}$. In other words $i_{mn}$ is the current extracted from the node at position $(N_x,n)$ when a current $i_0$ is injected at node $(0,m)$. We make the assumption that a current of $i_0$ is extracted from each node in the extraction electrode when a total current $I$ is injected into the injection electrode. The total extracted current must match the injected current and so
\begin{equation}
\sum_{n=0}^{N_y-1} i_{mn}=\sum_{m=0}^{N_y-1} i_{mn}= ~i_0
\end{equation}
The current equation in this case of a single injection node and multiple extraction nodes is
\begin{equation}
I(\vec{r})=I(x\vec{a}_1 + y\vec{a}_2) =  \delta(x)\delta(y-m) i_0 - \sum_{n=0}^{N_y-1} \delta(x - N_x) \delta(y-n)  i_{mn}
\end{equation}

By summing contributions of current $i_0$ injected at each point $(0,m)$ on the injection electrode, one can write the current equation of a system with two electrodes of width $N_y$ nodes separated by $N_x$ nodes as
\begin{equation}
I(\vec{r})=I(x\vec{a}_1 + y\vec{a}_2) = \sum_{m=0}^{N_y-1}\left( \delta(x)\delta(y-m) i_0 - \sum_{n=0}^{N_y-1} \delta(x - N_x) \delta(y-n)  i_{mn} \right)
\end{equation}
The voltage function from Eq.(\ref{potential_GF}) is now:
\begin{equation}
V(\vec{r}) = R \sum_{m=0}^{N_y-1}\left(G(\vec{r}-m\vec{a}_2)i_0 - \sum_{n=0}^{N_y-1}G(\vec{r} - N_x\vec{a}_1 - n\vec{a}_2) i_{mn}  \right)
\end{equation}
In order to calculate the resistance between the two electrodes, we calculate the difference between the average voltage on the electrode where current is injected and the average voltage on the electrode where current is extracted. We then divide by the total amount of current that was injected/extracted.
\begin{equation}
R_{e}(N_x) = \frac{1}{N_y I}\left(\sum_{b=0}^{N_y-1}V(0,b) - \sum_{b=0}^{N_y-1}V(N_x,b)\right)
\end{equation}
Consider finite electrodes in one of two systems. If the network is infinite is both $\vec{a}_i$ directions then one performs a continuous Fourier transform in both $\vec{k}$ directions for the greens function $G_i$. 
\begin{equation}
G_i(\vec{r} = x \vec{a}_1 + y \vec{a}_2) =   \int \frac{dk_1}{2 \pi} \frac{dk_2}{2 \pi} \frac{e^{i(k_1  x +  k_2 y)}}{2 - \cos(k_1) - \cos(k_2)}
\end{equation}
The 

Due to the finite size of the network in the $\vec{a_2}$ direction, $\vec{k_2}$ is discretised with values $k_2 = \frac{b}{N_y}, ~~b = 0,1,...,N_y-1$. The discrete Fourier transform and its inverse has been used in this direction. 
\begin{equation}
G(\vec{r} = x \vec{a}_1 + y \vec{a}_2) = \frac{1}{N_y}\sum_{b=0}^{N_y-1} \int \frac{dk_1}{2 \pi} \frac{e^{i(k_1  x +  \frac{2 \pi b}{N_y} y)}}{2 - \cos(k_1) - \cos(\frac{2 \pi b}{N_y})}
\end{equation}
Using the fact that $i_0 = I/N_y$ one can write the resistance equation as:
\begin{equation}
R_e(N_x,N_y) = R \int \frac{dk_1}{2 \pi} \sum_{b,l,m=0}^{N_y-1} \frac{1}{N_y^3}e^{\frac{i 2\pi b m}{N_y}}e^{\frac{-i 2\pi b l}{N_y}} \left(\frac{1 - e^{i k_1 N_x}}{2 - \cos(k_1) - \cos(\frac{2 \pi b}{N_y})} \right)
\end{equation}
One can single out a dirac-delta function using an inverse discrete  fourier transform
\begin{equation}
\frac{1}{N_y} \sum_{m=0}^{N_y} e^{\frac{i 2 \pi b m}{N_y}} = \delta(b)
\end{equation}
Subbing this in and summing over b
\begin{equation}
R_e(N_x,N_y) = R \int \frac{dk_1}{2 \pi} \sum_{l=0}^{N_y-1} \frac{1}{N_y^2} \left(\frac{1 - e^{i k_1 N_x}}{1 - \cos(k_1) } \right)
\end{equation}
The remaining integral can be performed using the residue theorem leaving us with
\begin{equation}
R_e(N_x,N_y) = R \frac{N_x}{N_y}
\label{finite_eqn}
\end{equation}
This seemingly complicated derivation has a straightforward interpretation. Consider chains of $N_x$ resistors of resistance R connected in series, and there are $N_y$ of these chains connected in parallel. 

The dependence of $R_e$ on $N_x$ and $N_y$ are explored using computational simulations in Figure \ref{fig:square_nx_ny_change}. Figure \ref{fig:square_nx_ny_change} a) presents the effect of changing the lead separation in a square lattice where $N_y = 30$ and each resistor in the network $R = 1 \Omega$. A clear linear relationship is evident between the two matching the analyical expression outlined in Equation \ref{finite_eqn}.  Figure \ref{fig:square_nx_ny_change} b) presents the effect of changing $N_y$ with $N_x = 30$ and fixed resistance $R = 1 \Omega$.
