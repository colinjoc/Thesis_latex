\chapter{Summary and Future Work}

In this thesis, many aspects of the electrical transport properties of nanowire networks and their dependence on underlying nanowire parameters have been examined. The main goal of this thesis was to develop mathematical models and computer simulations to succesfully capture these transport properties of NWNs, and to explain and predict experimentally measured responses of NWNs. Networks were studied with two main inter-nanowire junction elements; where junctions have a static, and a dynamic response to electrical stimulus. Here a summary is presented that reiterates the key points of this thesis as well as highlighting possible extensions to this body of work.
%================ 1 ======================
\section{Thesis Summary}
In chapter 1, a general overview to the thesis was presented that provided context to the material that was reported in the following chapters. Here, an introduction to nanowire networks was given, and some of their many cutting-edge applications were discussed; from efficient transparent conductors to a highly connected networks of memristive elements for neuromorphic devices. Mathematical techniques and models that are used to accurately analyse aspects of NWNs in this thesis were introduced. The main goal of the thesis was outlined here and motivated the theme of the reported research to follow.

%================ 2 ======================
Chapter 2 details the necessary background theory and mathematical formalisms to understand current-flow through a nanowire network. It was shown that the electrical properties of a network can be calculated by solving a system of linear equations containing the connectivity profile and resistances of the network corresponding to Kirchhoff's circuit law. Then, an analytical method to calculate resistances in an ordered infinite lattice was discussed. An effective medium theory for ordered resistive lattices was then derived, and was applied to a two dimensional square lattice with a bi-modal resistance distribution. A brief introduction to percolation theory was also given, in particular the critical wire density for a conductive stick system in two dimensions was discussed. Finally a functional form for the number of junctions in a NWN was derived that relates the number of junctions with the number of wires and wire lengths.  

%================ 3 ======================

In chapter 3, a method to calculate equivalent resistances in a nanowire network using the Kirchhoff set of linear equations were introduced; the Junction Dominated Approach (JDA) and the Multi-Nodal Representation (MNR). The JDA model only considers nanowire junctions as resistive elements, the MNR model also includes a contribution of inner nanowire resistance. Using both models, the inclusion of inner-wire resistances was shown to significantly alter the dependence of the sheet resistance on various network parameters compared to when only junction resistances are considered. Sheet resistances were shown to have a linear dependence on nanowire junction and inner nanowire resistances for both MNR and JDA models. Then, a power law relationship between sheet resistance and wire density was observed as one would expect from percolation theory in simulations of ensembles of NWNs. The inclusion of inner nanowire resistance was shown to alter the value of the exponent in the percolative power laws. Following this, a method to capture the geometrical layout of a physical NWN sample from an SEM image was presented, and was used to simulate NWNs on geometries similar to an experimental sample. These simulations were used to identify characteristic junction resistances for annealed Ag/PVP nanowires, estimating the junction resistances to be of the order of tens of Ohms, which was later experimentally confirmed by Bellew et al \cite{bellew2015}. Then, the ultimate conductivity of a network was shown to be limited by the contribution of the inner nanowire resistances. To understand how much potential for conductivity improvement a network has, the optimisation capacity coefficient ($\gamma$) was defined to capture this, and was shown to depend on the characteristic junction resistance of experimental samples. Finally, the impact junction resistance dispersion has on the sheet resistance of a NWN was demonstrated. It was shown that dispersion can break the linear relationship between sheet and junction resistance, which can shift the characteristic junction resistance to lower values compared to estimates obtained with a homogeneous resistor distribution.

%================ 4 ======================

In chapter 4, an effective medium theory was used to establish a mapping between the sheet resistance of a heavily disordered NWN with that of an ordered square lattice. To achieve this, expressions for the relative percentages of types of resistors in a NWN were derived. These expressions can be used to determine a quantitative critical wire density at which a percolating path between electrodes is impossible, which is much less than the one suggested by percolation theory. The expressions for relative resistor percentages were then used to create an effective medium, one which maps the resistive properties of a random NWN onto a regular square lattice. This mapping was used to approximate the equivalent resistance for different nodal separation between pairs of junction intersections in a random NWN, and was shown to match simulations closely. The sheet resistance of a finite square lattice where two opposite sides are bounded by electrodes was shown to be caused by several identical paths of resistors connected in series. To apply this behaviour and the effective square lattice mapping to a NWN, expressions to approximate the number of parallel paths and their lengths were derived, and gave reasonable estimates when compared with simulations. After combining the effective square lattice with the expressions for its size, a closed-form approximation for the sheet resistance of a NWN in terms of its underlying geometrical and electrical properties was then derived. The effective lattice was shown to approximate the sheet resistance well for changing fundamental parameters, the scaling between sheet resistance and wire density as calculated by the effective lattice agreed closely with simulations. The effect of junction resistance and wire resistivity on sheet resistance was also estimated with the effective lattice and gave similar results to the simulations. The effective lattice was then applied to thirty experimental samples and used to gauge the percentage of high resistance junctions that were present in the samples. The effective lattice was also used to estimate the ultimate conductivity of a network which was then used to calculate the optimization coefficient instantaneously, a calculation that comprises of several simulations when performed with the MNR model.

%================ 5 ======================
The memristive response of nanowire junctions to increasing current flow through a NWN was modeled using a bottom-up approach in chapter 5. This was achieved by describing the individual junction response as a power-law plus cut-offs (PL+C) and using this to simulate the collective response of a network of such junctions. The PL+C model was developed based on experimental measurements of nanowire junctions which showed a power law relationship between their conductance and the current compliance. A self-similar scaling between the conductance evolution of a NWN and an individual junction was found in simulations, confirming experimental measurements of NWNs where junctions had measured exponents $\alpha_j \approx 1$. The network memristance was shown to have three main scaling dynamics depending on the value of the junction scaling exponent $\alpha_j$, whether it is sub-linear, linear or supra-linear. In all three cases, a self-similar power law was identified between junctions and networks. In other words, the network was found to scale as a power law with an exponent similar to that of the junctions. For supra-linear junctions, the emergence of highly conductive paths that display a winner-takes-all behaviour was shown. Evidence for this was the appearance of a steady-state of the NWN conductance for increasing current-flow, where the network entered a period of inactivity at a conductance corresponding to a single path of fully evolved junctions. Current colour mappings were introduced as a means to visualise the current-flow through a NWN, and showed that the winner-takes-all paths did indeed emerge. The emergence of winner-takes-all paths could be used in devices for neuromorphic applications, and so a multi-electrode architecture for a NWN was designed as a proof-of-concept. The device had several addressable inter-electrode paths that could be interrogated while one of the paths was driven to a high conductance state. Simulations suggest that several independent addressable memory states could be stored in a NWN, and point to a process to achieve associative memory states in a NWN through the use of shared electrodes.

%================ 6 ======================

A model that describes nanowire junctions as a binary state capacitor that transitions to an activated state once a critical potential difference across the junction was introduced in chapter 6. The capacitive model (CPM) was contrasted with the memristive model (MRM) outlined in chapter 5.  The two successfully model the capacitive and memristive responses of NWNs at distinct transport regimes; the capacitive to extremely low current levels ($\sim$ pA) and the memristive to currents in the range of nA-$\mu$A. By applying both models to an identical network geometry, the contrasting dynamics of both models were highlighted. Following this, the CPM displayed a network-wide activation pattern before a continuous path of activated junctions existed between electrodes. This contrasts with the highly localised path that evolved in supra-linear junctions in the MRM. The CPM was shown to move between periods of idleness to abrupt cascades of mass activations of junctions in the network as the potential difference across the device was increased. The size frequency distribution of these activation events, or avalanches of activations, were shown to follow a power-law relationship which is indicative of scale-free complex network dynamics, i.e. the effect of perturbations to the network is only limited by the size of the network. Experimental evidence for the complex dynamics of a physical nanowire network with negligible current-flow was presented and corroborates the scale-free nature captured by the CPM simulations. Finally, the fault-tolerance of NWNs modeled with CPM and MRM were probed by analysing the response of the network to a junction failure in the main inter-electrode path. It was found that networks modeled with the MRM were very robust with respect to fault-tolerance. Sheet conductances were slightly perturbed with the failure of a junction in the WTA path for supra-linear junction exponent simulations. On the other hand, NWNs modeled with CPM were shown to be very sensitive to junction failure. A junction failure greatly perturbed the activation dynamics of the network, more than double the junction activations to reform a shorting path between the electrodes were required after the junction failure. 

%============================
\section{Future Work}
In this thesis, many aspects of the resistive properties of nanowire networks have been presented. The discussion included networks with static internal parameters, i.e. those that did not change in response to electrical perturbations. Dynamic responses of networks were also presented, a memristive and capacitive junction response manifested a rich range of physical phenomena in nanowire networks. Both static and dynamic features in the NWN systems have a vast potential to yield fascinating results from future investigations. In this section, some future investigations that naturally follow the results presented in this thesis are discussed.

First we shall address the effective medium lattice that was developed in chapter 4. There are several investigations that follow from this approach. Very recently, He et al reported an effective medium model for NWN sheet resistance\cite{he2018} similar to that derived in chapter 4. In their manuscript, He et al removed clusters of electrically inactive nanowires from the effective medium description and described the relationship between sheet resistance and the underlying network parameters. The removal of unconductive sections was performed numerically in the simulations and corresponded to clusters of nanowires isolated from the main conducting clusters and sections where only one electrical connection to the cluster existed, thus creating a dead-end for current flow. The non-conducting sections are more prevalent at low wire densities according to He et al; at low densities the effective square lattice approximation tended to underestimate the sheet resistance in chapter 4. By extending the first principles derivation of the three types of resistors that occur in a NWN that was presented in chapter 4, a correction that accounts for non-conducting clusters of the network will improve the accuracy of our model, particularly at sufficiently low densities.

In chapter 1, an overview of the many applications of nanowire networks was given, in particular their application as transparent conductors. Many of these depend on nanowire networks not only having specific conductivities but transparencies as well. Several models have been succesfully applied to linking transparencies and sheet resistances, however most are empirically based and require experimental measurements to properly link the two\cite{chung2012,selzer2016,bergin2012,mutiso2013}. Very recently, Ainsworth et al\cite{ainsworth2018} reported a refined relationship between sheet resistance and transparency that incorporated inner-wire resistances and inter-wire junction resistance contributions as fitting parameters. Ainsworth et al demonstrated the accuracy of their model by succesfully fitting it to reported measurements found in the literature\cite{ainsworth2018}. Mie scattering theory was applied to nanowire networks by Khanarian et al\cite{khanarian2013} and Ainsworth et al\cite{ainsworth2018}, and used to calculate the transmission of a network as a function of wire diameter, and surface fraction. By associating this analytic theory with the effective medium lattice derived in chapter 4, a closed form expression linking network transparency with sheet resistance could be derived. Such an expression would be a major aid to developing industrial applications of nanowire networks as it would incorporate all of the relevant nanowire parameters, including inner-wire resistance, which as we saw in chapters 3 and 4 plays a vital role in network conductivity.

In chapter 5, the memristive properties of nanowire networks was presented. This property in particular offers a rich new area for future research in nanowire networks. One of the more immediate projects is to apply the analytical models developed for resistive switching in ECM and VCM devices to a network of such elements, as is the case in nanowire networks. There are models in the literature that captures the dynamics of migrating ions in VCM and ECM cells; these have been shown to accurately reproduce the behaviours of their experimental counterparts\cite{hansen2015,menzel2015,menzel2017}. A method to incorporate these models with Kirchhoff's system of linear equations must be developed to simulate complex networks of such junctions. Such nanoscale models will enable a quantitative description of many of the device parameters relevant to memristive and ReRAM devices in NWNs such as the SET voltage, and switching speed. A further addition to be made to a nanoscale model of junction memristance is to include a memcapacitive element\cite{wakrim2016}. A memcapacitor has a tunable capacitance that is mediated by some internal physical property of the system. In nanowire junctions, this could be mediated by varying conductive filament length, thus coupling the memristive response of a device with its capacitance. This extension to a nanoscale junction model would allow a nanowire network to be modeled at leakage current levels as well as conductive filament growth levels.

In chapter 5, a multi-electrode device that could facilitate multiple winner-takes-all pathways in a nanowire network was presented. Associative and independent states were demonstrated in a proof-of-concept network by exploiting the $2 \times 2$ electrode layout. This property needs to be further explored and an understanding as to how a dependence on wire length and density affects these properties. Similarly new electrode architectures are to be developed that further utilise these properties to create memory devices or even logic elements. Central to this process, simulated nanowire network devices of various material properties and electrode architectures will play a vital role in designing neuromorphic applications. 

 


