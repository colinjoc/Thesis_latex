\chapter{Effective Medium Theory for Nanowire Networks}

A difficulty with NWN adaptability for use in devices at an industrial scale is their random nature. The random connectivity profile, and varying junction resistances and wire resistivity all conspire to make the fabrication of a NWN with a desired sheet resistance difficult. Two NWNs comprised of identical wires and of similar wire densities can have wildly different electrical properties due to the inherent network disorder, frustrating reproducibility in experiments. In simulations, this large variance is caused by two main types of disorder: the randomness with which wires are spatially distributed\cite{ kallmes1960, sampson2008} and the inherent fluctuations on the characteristics of the individual wires. This calls for averaging strategies that reduce the impact of these fluctuations in any calculations, requiring a large amount of computational resources to determine the average sheet resistance for fixed nanowire properties, as was seen in chapter 3. With that in mind, a novel method that processes Scanning Electron Microscope (SEM) images of NWNs and captures the precise locations of all wires of a given sample was presented in chapter 3. An example of an SEM image is presented in Figure \ref{fig: nwn_mapping_sketch}(a), and its digitised form is shown in Figure \ref{fig: nwn_mapping_sketch}(b). This establishes a benchmark connectivity the NWN possesses and removes the need for averaging over the wire locations, consequently reducing the fluctuations induced by spatial disorder. A matching connectivity profile for experiment and simulations allows for an extraordinary level of simplification in simulation and experiment comparisons. A further simplification can be made through a theoretical description of the electrical properties of a network giving a method to quickly estimate the characteristics of a real NWN sample. In this chapter, such an expression is succesfully derived for a NWN by developing a mapping to an effective medium lattice whose sheet resistance can be calculated with a closed-form expression\cite{ocallaco2016}.

Contemporary theoretical descriptions of real-world NWNs in which their physical characteristics are accounted for is typically done by means of robust Monte-Carlo procedures used to determine universal behaviours of simplified computer-generated NWNs \cite{pike1974,li2009,zezelj2012,ashkan2007,khanarian2013,ainsworth2018}. However, a method to obtain a closed-form analytical expression for the conductance of infinite ordered homogeneous networks is known\cite{cserti2000}, and was discussed in chapter 2. Whether this method can be of use to describe heavily disordered structures, even though they are far from ordered, is the strategy adopted in this chapter. By mapping the disordered structures onto a corresponding effective square lattice that was discussed in chapter 2 for instance, we can obtain the sheet resistance of a NWN with an arbitrarily large density of wires. This mapping is visualised in Figure \ref{fig: nwn_mapping_sketch} where the resistive network graph in panel (c) will be discussed as an effective square lattice shown in panel (d). Further manipulation of these representations enables us to describe the conductivity of these films under real experimental conditions. In fact, we show that dense networks composed of nanowires of non-uniform lengths and diameters contacted by finite-sized electrodes can be described by this approach\cite{ocallaco2016}.

\fig{0.8}
{Images/Chapter4/nwn_network_sketch_hq.pdf}
{\textbf{Sketch:} The steps of mapping a NWN image onto an effective lattice.}
{(a) SEM micrograph image of a Ag/PVP NWN with hundreds of wires randomly distributed on top of an insulating substrate. Two electrodes on both sides of the sample, shown as vertical gray bars, are connected by numerous paths formed by the wires. (b) After the image is processed, the digitized version of the image records each wire location and provides full information about the intersection points of each wire\cite{rocha2015}. (c) A mathematical graph of a digitised network such as that in panel (b) showing voltage nodes as points and connecting resistors as edges. (d) The simplified graph of a square lattice representing a regular ordered network\cite{ocallaco2016}. }
{fig: nwn_mapping_sketch}

The sequence of this chapter is as follows. In section \ref{Sec: Inter node}, the equivalent resistance between two nanowire junctions in a random NWN is shown to behave in a similar manner to the equivalent inter-nodal resistance in an infinite square lattice that was discussed in chapter 2, suggesting that a mapping between the two systems can be established. An Effective Medium Theory (EMT) is introduced for regular lattices, and is then used to determine this mapping between a square lattice and a random NWN in terms of the underlying properties of the constituent nanowires in section \ref{Sec: NWN EMT}. Following this, the resistance for a multi-input/output electrode system is discussed, in particular for a system where the electrodes span opposite sides of a finite square lattice in section \ref{Sec: Inter Electrode}. By combining the effective lattice mapping for a NWN with the multi-electrode resistance expression, a closed-form expression is successfully derived for the resistive properties of a NWN based on all of the relevant nanowire properties. This expression is then used to determine various parameters and properties of NWNs that were discussed in chapter 3; this is presented in section \ref{Sec: EMT Application}. Finally a chapter summary is presented in section \ref{Sec: EMTGF Conclusion}. 

%==============================================
\section{Inter-nodal Resistance in a Nanowire Network}
\label{Sec: Inter node}
In chapter 2, an expression for the equivalent resistance between two nodes in an infinite resistor lattice was derived using the Green's function method\cite{cserti2000}. An approximation to the Green's function was derived and shown to be highly accurate for increasing separation between nodes in the lattice. Where two nodes are separated by the lattice vector $\vec{r}$, the approximation for the inter-nodal resistance is\cite{cserti2000} 
\begin{equation}
R_{eq}(\vec{r}) \approx \frac{R}{\pi} \left( \ln(|\vec{r}|) + \gamma + \frac{\ln(8)}{2} \right) 
\label{eq: Rnn_approx}
\end{equation}
where $R$ is the resistance of each edge in the lattice and $\gamma = 0.5772...$ is the Euler-Mascheroni constant. Equation \ref{eq: Rnn_approx} was shown to match the equivalent resistance calculated numerically for a finite square resistive lattice using Kirchhoff's and Ohm's laws in chapter 2. With an understanding of the electrical properties of inter-nodal currents in ordered resistive networks, the question arises if current flow in NWNs behaves in a similar manner. In Figure \ref{fig:inter_node_nwn_sketch}, a sketch of several nanowires is shown with two electrode nodes represented by red dots. For an accurate comparison with a resistive lattice, the distance metric used will be the nodal distance, or the number of resistors in the shortest path between two electrodes. Figure \ref{fig:inter_node_nwn_sketch} is a sketch of a NWN where the shortest nodal path between the two electrodes is depicted with blue arrows. The path contains three wire segments and one inter-wire junction making the nodal separation equal to four.  

\fig{0.75}
{Images/Chapter4/inter_node_sketch_proper.pdf}
{\textbf{Sketch:} Visualisation of nodal separation.}
{A sketch of a NWN with nanowires represented as grey cylinders, and the source and drain electrodes as red circles. The shortest path between electrodes is traced with the blue arrows through a single nanowire junction represented as a black circle and three wire segments, making the nodal separation between the electrodes equals four.}
{fig:inter_node_nwn_sketch}

A large NWN with no bounding electrodes was simulated to calculate the relationship between $R_{eq}$ and the nodal separation for an ensemble of pairs of nodes, using the MNR node-voltage mapping to include inner-wire resistances. The simulated nanowires were all of length $L =7 ~ \mu m$, junction resistance $R_j = 11 ~ \Omega$, wire resistivity $\rho = 22.6 ~ n \Omega m$, and wire diameter $D = 60 ~ nm$. The average inter-nodal resistance for a given nodal separation is shown in Figure \ref{fig:free_inner_res}. Immediately one identifies that the resistance between nearest neighbours is relatively large and uncertain. This large variance is due to the nearest neighbour being either a junction or an inner-wire segment which itself has a large variance as it depends on the length of that wire segment. The uncertainty decreases for increasing nodal-separation due to the fluctuations of inner-wire resistance values and junction resistances being averaged over. At larger separations, the uncertainty grows once again. This is likely due to finite sized effects as one or both of the nodes begin to reach the edge of the network. There is an unmistakable trend between $R_{eq}$ and the node separation which appears log-like but the large error bars in the data points might obscure this. The green line in Figure \ref{fig:free_inner_res} corresponds to a fit of equation \ref{eq: Rnn_approx}, where the effective resistance was the fitting parameter and found to be $R = 6.09 ~ \Omega$. This suggests that the inter-nodal resistance in a random NWN can be approximated using an expression derived for a regular square lattice, requiring only an appropriate effective resistance that represents both the wire segment and junction resistances. A method to calculate an effective resistance is outlined in the following section. 

\fig{1}
{Images/Chapter4/free_inner.pdf}
{\textbf{Plot:} Equivalent resistance between nodes in a NWN.}
{ The inter-nodal resistance ($R_{eq}$) for a given nodal separation in a random NWN. The resistance between nodes is calculated using the MNR model which was described in chapter 3, thus including the effect of inner-wire resistance. The red points correspond to the average $R_{eq}$ for a given node separation and the error bars are 95\% confidence intervals. The green line is the fitting of equation \ref{eq: Rnn_approx} where $R = 6.09 ~ \Omega$ is the fitting parameter.}
{fig:free_inner_res}
%=========================================================================
\section{Effective Medium Theory of a Nanowire Network}
\label{Sec: NWN EMT}
In chapter 2, the Effective Medium Theory (EMT) for conduction in resistive lattices\cite{kirkpatrick1973} was introduced where a network whose conductor values $g$ follow some distribution $f(g)$ and can be mapped onto a homogeneous network that has the same average properties. The effective resistance is calculated such that the average resistive properties of the homogeneous and inhomogeneous networks are the same. An effective medium theory requires an understanding of the resistor distribution $f(g)$, both the resistance values and relative proportions of each. With several assumptions, patterns can be identified in the types of resistors that make up a NWN. In this section, the different types of resistors in a NWN are established and analytical expressions to calculate their relative populations are derived. 

In the MNR voltage mapping each node has three edges connected to it, one junction resistor and, either two inner-wire segments or a single wire segment and a `dead-end'\cite{ocallaco2016}. Dead-ends occur at either end of a nanowire (two per nanowire), and are represented by an infinite resistance connection $R_d \rightarrow \infty$. Mathematically, the number of dead-ends in a network is $N_d = 2 N_w$, $N_w$ being the total number of nanowires. The number of junction resistors is $N_j$ and each of their values will follow some junction resistance distribution $\sigma_j(R_j)$. Finally there are current carrying inner-wire segments whose relative percentages can be derived with the following logic: consider a network with no inter-wire junctions; the number of wire segments is clearly the number of wires in the network. For every inter-wire junction that is added to the network, two wire segments are formed (one on each wire). Thus the total number of wire segments $N_s$ is
\begin{equation}
N_s = N_w + 2 N_j
\end{equation}
This expression for the number of segments includes dead-ends, and so the number of current-carrying wire segments is 
\begin{equation}
N_{cc} = N_s - N_d = 2 N_j - N_w
\end{equation}
Figure \ref{fig:nwn_emt_sketch} presents a sketch of a simple NWN that identifies the different types of resistors. In this network, $N_w = 5$ and $N_j = 5$, and so there are $N_d = 10$ dead-ends which are coloured in blue and there are $N_{cc} = 5$ current carrying segments coloured in red.

\begin{comment}
\fig{1}
{Images/Chapter4/simple_nwn_component_final.pdf}
{\textbf{Sketch:} Types of resistors in a NWN.}
{(a) A sketch of a NWN and (b) the different types of resistors highlighted in the network. In this network, there are five wires ($N_w = 5$) and the five junctions ($N_j = 5$) depicted by black circles. There are ten dead-ends ($N_d  = 10$) and these are depicted by blue segments. There are five current-carrying segments, $N_{cc} = 2 N_j - N_w$, shown in red. In total there are $N_t = N_j+ N_d + N_{cc} = 20$ resistors in this network.}
{fig:nwn_emt_sketch}
\end{comment}
\fig{0.75}
{Images/Chapter4/simple_nwn_component.pdf}
{\textbf{Sketch:} Types of resistors in a NWN.}
{A sketch of the different types of resistors in a NWN, where each is highlighted. In this network, there are five wires ($N_w = 5$) and the five junctions ($N_j = 5$) depicted by black circles. There are ten dead-ends ($N_d  = 10$) and these are depicted by blue segments. There are five current-carrying segments, $N_{cc} = 2 N_j - N_w$, shown in red. In total there are $N_t = N_j+ N_d + N_{cc} = 20$ resistors in this network.}
{fig:nwn_emt_sketch}


The total number of conductors $N_t$ is the sum of inter-wire junctions $N_j$ and wire segments including dead-ends as
\begin{equation}
N_t = N_j + N_{cc} + N_d = 3 N_j+N_w
\end{equation}
The relative percentages of each type of resistor is thus
\begin{eqnarray}
P_j &=& \frac{N_j}{3 N_j+N_w} \nonumber \\
P_{cc} &=& \frac{2 N_j - N_w}{3 N_j+N_w} \nonumber \\
P_d &=& \frac{2 N_w}{3 N_j+N_w}
\end{eqnarray}
where $P_j$ is the percentage of junction resistors, $P_{cc}$ the percentage of current carrying wire segments, and $P_d$ the percentage of dead-ends. These relative populations can be easily translated into expressions in terms of junction and wire densities by dividing both numerator and denominator by the area of the network. In chapter 2, an expression relating the junction density with the wire density of a nanowire network with wires of length $L$ was derived as\cite{kallmes1960,sampson2008,ocallaco2016} $n_j = \omega L^2 n_w^2$ where $\omega = \pi^{-1} \approx 0.318$. Following this expression the relative percentages in terms of wire densities and their lengths are given by
\begin{eqnarray}
P_j &=& \frac{ \omega L^2 n_w}{3 \omega L^2 n_w + 1} \nonumber \\
P_{cc} &=& \frac{2 \omega L^2 n_w - 1}{3 \omega L^2 n_w + 1} \nonumber \\
P_d &=& \frac{2 }{3 \omega L^2 n_w + 1}
\label{eq: resistor_percentage}
\end{eqnarray}
Equations \ref{eq: resistor_percentage} allow one to calculate the population of each type of resistor in a NWN knowing only the total number of wires and their length. This is a more desirable form for the relative percentages of populations as it does not require one to explicitly count the number of junctions in a NWN. The nanowire density and typical wire lengths are predefined parameters in typical Monte Carlo simulations and are straightforward to measure in physical NWN samples. An important note with respect to equations \ref{eq: resistor_percentage} is that they hold for networks where all wires are of length $L$ as this is a condition for the calculation of $n_j$. 

\fig{1}
{Images/Chapter4/relative_percentages.pdf}
{\textbf{Plot:} EMT relative percentages of resistors in a NWN.}
{A plot of the relative percentages of the different types of resistors in a NWN as a function of wire density for a NWN with wire lengths of $7\mu m$. The red curve is the percentage of current-carrying wire segments $P_{cc}$, the black curve is the percentage of junctions $P_j$ and the green curve is the percentage of dead ends $P_d$. The vertical purple dashed line is the percolative critical wire density $(n_w)_c$ at which a percolative path occurs in 50\% of randomly generated networks with this density\cite{li2009}.}
{fig:emt_nwn_percentages}

Figure \ref{fig:emt_nwn_percentages} presents a visualisation of these relative percentages as a function of wire density for a NWN with nanowires of length $7 ~ \mu m$. Note that as the wire density tends to infinity, the percentage of dead-ends tends to zero, while the number of junctions tends to 1/3 and the number of current-carrying segments tends to 2/3. As mentioned previously, each node in MNR has three nearest neighbours and is connected to a junction resistance and either two-current carrying wire segments or one current-carrying wire segment and a dead end. As the number of wires tends to infinite, the percentage of dead-ends drops to zero and so the percentages tend to the 1/3 junctions and 2/3 current-carrying segment percentages. On the other extreme, a critical wire density of sorts can be identified at which the percentage of current-carrying segments is zero according to the definition of $P_{cc}$ in equation \ref{eq: resistor_percentage}.
\begin{equation}
(n_w)_0 = \frac{1}{2 \omega L^2}
\end{equation}
For wire lengths of $L = 7 ~ \mu m$, this gives $(n_w)_0 \approx 0.035$ nanowires/$\mu m^2$. At this value there are only dead-ends and wire junctions which does not result in a conductive network as there are no conducting wire segments through which current can flow. $(n_w)_0$ is the minimum wire density that is considered in Figure \ref{fig:emt_nwn_percentages} as below this density $P_{cc}<0$.

The population of each type of resistor is only one of the components to the full conductance distribution $f(g)$. One also requires the distribution in conductance values of each type of resistor. Recall from chapter 2 that the effective conductance $g_m$ is calculated using the following equation\cite{kirkpatrick1973} 
\begin{equation}
\int \frac{g_m-g}{g+(z/2)g_m} f(g) ~ dg = 0
\label{eq:emt_def}
\end{equation}
where $z$ is the degree of each node in the lattice. In general, the distribution for NWNs to be used in equation \ref{eq:emt_def} is:
\begin{equation}
f(g) = P_{cc} \sigma_{cc}(g) + P_j \sigma_{j}(g) + P_d \delta(g) 
\label{eq: long_nwn_conductance_dist}
\end{equation}
where $\sigma_{cc}(g)$ is the distribution of inner-wire conductances, $\sigma_{j}(g)$ is the distribution of junction conductances, and since all dead-ends have a conductance $g=0$, its distribution is characterised by the Dirac delta function $\delta(g)$. While the populations of each type of resistor have been given in equation \ref{eq: resistor_percentage}, the distributions to be used for the junction and inner-wire resistances have not. The junction resistances in Monte Carlo simulations are usually fixed to some homogeneous value $g_j$ and so $\sigma_{j}(g) = \delta(g-g_j)$. The conductance of a current carrying inner-wire segment is given by $g_{cc} = \frac{A_c}{\rho \ell}$ where $\rho$ is the resistivity, $A_c$ the cross sectional area, and $\ell$ the length of the wire segment. The inner-wire conductance distribution will be approximated by the average length of a wire segment $\tilde{l}_s$, and is calculated by dividing the total length of all the wires by the number of wire segments in the network. 
\begin{equation}
\tilde{l}_s= \frac{L N_w}{N_s} = \frac{L N_w}{N_w + 2 N_j}
\end{equation}
where $L$ is the length of each wire. It follows then that the characteristic inner-wire conductance is $g_{cc} = A_c/\rho \tilde{l}_s$ making the conductance distribution $\sigma_{cc}(g) = \delta(g - g_{cc})$. Equation \ref{eq: long_nwn_conductance_dist} simplifies to
\begin{equation}
f(g) = P_{cc} \delta(g - g_{cc}) + P_j \delta(g - g_j) + P_d \delta(g)
\label{eq: simple_nwn_conductance_dist}
\end{equation}
This approximation of the conductance distribution in NWNs can now be used to solve for the effective conductance of an ordered square lattice. Solving equation \ref{eq:emt_def} for a square lattice (degree $z = 4$) and with the distribution given in equation \ref{eq: simple_nwn_conductance_dist} the effective conductance is given by\cite{ocallaco2016}
%\begin{align}
%g_m &= \frac{g_j (N_w - 3 N_j) - 2g_{cc} N_w}{6N_j - N_w} +\frac{1}{6N_j N_w} \Big(9g_{cc}(8g_{cc} + g_j) N_j^2 - \nonumber \\
%& - 6g_{cc} N_j N_w(8g_j+g_{cc}) +(2g_{cc}+g_j)^2N_w^2 \Big)^{\frac{1}{2}} \label{Eq:gm_nwn_original}
%\end{align} NEW
\begin{align}
g_m &= \frac{g_{cc}(N_j - 3 N_w) - g_j(N_j+N_w)}{6 N_j + N_w} + \frac{1}{6 N_j + N_w} \times \nonumber \\
&\times \Big( 12 g_{cc} g_j (N_j - N_w) (3 N_j + N_w) + (g_{cc} (N_j - 3 N_w) - g_j (N_j + N_w))^2 \Big) ^{1/2}\label{Eq:gm_nwn_original}
\end{align}
Rewriting the number of junctions in terms of the wire density using the relationship $n_j = \omega L^2 n_w^2$, the effective conductance can be written as\cite{ocallaco2016}
\begin{eqnarray}
g_m = &\frac{g_{cc}(\omega L^2 n_w -3) - g_j (1 + \omega L^2 n_w)}{2+6\omega L^2 n_w^2} +\frac{1}{2+6\omega L^2 n_w} \Big( 12 g_{cc} g_j(\omega L^2 n_w -1)(1+3\omega L^2 n_w) \nonumber +\\
&+ (3 g_{cc} +g_j +\omega(g_j-g_{cc})L^2 n_w)^2 \Big)^{1/2} 
\label{eq:gm_nwn_wiredens}
\end{eqnarray}
Recalling the definition of the characteristic inner-wire resistance, equation \ref{eq:gm_nwn_wiredens} is an expression in terms of the wire length, density, diameter, resistivity, and junction resistance which are all predefined parameters of a NWN. This means simulations are not required to calculate the parameters for the effective conductance.

Revisiting the inter-nodal resistance for a NWN shown in Figure \ref{fig:free_inner_res}, the effective resistance calculated using equation \ref{eq:gm_nwn_wiredens} and parameters matching those of the simulated network is $R_{EMT} = 6.01~\Omega$. Recall that in Figure \ref{fig:free_inner_res}, equation \ref{eq: Rnn_approx} was fit to the simulation data with the resistance $R$ as the only fitting parameter resulting in $R_{fit} = 6.09~\Omega$. Figure \ref{fig:nwn_emt_log} presents the data shown in Figure \ref{fig:free_inner_res} alongside the approximation to the lattice Green's function given by equation \ref{eq: Rnn_approx} with the effective resistance found through regression as the green line, and the effective resistance calculated using equation \ref{eq:gm_nwn_wiredens} is the blue dashed line. It should be reiterated here that no fitting parameter was used in calculating $R_{EMT}$. The agreement between $R_{fit}$ and $R_{EMT}$ is remarkably close considering that mapping the NWN onto an effective medium square lattice involved several assumptions in creating the resistance distributions that represent the disordered nature of the network in an effective way.

\fig{1}
{Images/Chapter4/free_inner_emt.pdf}
{\textbf{Plot:} Inter-nodal Resistance EMT }
{Data points correspond to the inter-nodal resistance that was shown in Figure \ref{fig:free_inner_res}. The green solid line was obtained by fitting equation \ref{eq: Rnn_approx} to the data with the resistance $R$ the only fitting parameter which was calculated as $R_{fit} \approx 6.09 \Omega$. The blue dashed line represents equation \ref{eq: Rnn_approx} with $R = R_{EMT} = 6.01 \Omega$ calculated using equation \ref{eq:gm_nwn_wiredens}. }
{fig:nwn_emt_log}

%=======================================
\section{Inter-Electrode Resistance in a Nanowire Network}
\label{Sec: Inter Electrode}
In the previous section, the inter-nodal resistance in a NWN was successfully calculated by mapping its resistive properties onto an effective medium square lattice. As seen in Figure \ref{fig: nwn_mapping_sketch}(a), NWNs are usually fabricated with bounding electrodes on two opposite sides in order to measure the sheet resistance. For a particular NWN, let $H$ be the height of the electrodes and let $W$ be the separation between the electrodes. In this section the mapping between NWNs and the effective medium square lattice is extended to take into account the bounding electrodes such as those represented as red vertical lines in the sketch of a NWN in Figure \ref{fig:interElectrode_square_nwn}(a). The first step for this is to calculate the resistance of a finite homogeneous square lattice that is bounded on either end by an electrode, such as the system sketched in Figure \ref{fig:interElectrode_square_nwn}(b), where the electrodes are represented $N_y=7$ nodes (black square)  separated by $N_x = 13$ resistor edges. Electrodes are at equipotential and so in a homogeneous network, by symmetry, each column of nodes are also at equipotential which varies as one moves from left to right. In this scenario, no current flows between nodes in the same column due to there being no difference in potential and so the square lattice can be viewed as $N_y$ parallel paths each containing $N_x$ resistors in series. The inter-electrode resistance $R_e$ is then given by
\begin{equation}
R_e = R \frac{N_x}{N_y}
\label{eq: inter-electrode}
\end{equation}
where $R$ is the resistance of each network edge. Equation \ref{eq: inter-electrode} can also be determined by generalising the Green's function method outlined in chapter 2 to a finite square lattice where two opposite network boundaries are completely spanned by electrodes.

Figure \ref{fig:interElectrode_square_nwn}(a) presents a sketch of a NWN where there are 7 nanowire intersections with the electrodes on each side and 13 total resistors (including both junction and inner-wire resistors) in the shortest path between the electrodes. The shortest paths between electrodes are determined by applying a path finding algorithm to the graphical representation of the NWN\cite{yen1970}; it is the same method used in the previous section to calculate the nodal separation. Figure \ref{fig:interElectrode_square_nwn}(b) is a mapping of the NWN in panel (a) onto a square lattice that has the same graphical dimensions as the NWN. In order to compare the mapping between a NWN and a square lattice with extended electrodes, the dependence of $R_e$ on $N_x$ was identified by simulating a large NWN and calculating $R_e$ with electrodes placed at various $N_x$ separations, keeping the number of electrode intersections at $N_y = 7$. The relationship between $R_e$ and the calculated nodal separation $N_x$ is plot in Figure \ref{fig:interElectrode_square_nwn} (c) for the NWN and its effective square network with a matching $N_x$ and $N_y$. A clear linear relationship exists for the effective medium square lattice which is represented by the green triangles whereas $R_e$ fluctuates around the linear trend for the NWN. The effective medium square lattice approximates the simulated NWN very well. In this example the quantities $N_x$ and $N_y$ were explicitly calculated for the NWN shwon in Figure \ref{fig:interElectrode_square_nwn}(a), which is considered a relatively small network. This process can become a quite intensive calculation process for large and dense networks. An analytical method to calculate $N_x$ and $N_y$ would remove the necessity of simulations by providing a complete description of NWN sheet resistance formulated in a closed-form expression.

\fig{0.75}
{Images/Chapter4/interElectrode_square_nwn.pdf}
{\textbf{Plot:} Inter-electrode resistance versus electrode nodal separation.}
{(a) A simulated NWN with two separate finite-sized electrodes represented by vertical red lines of length $H$ and a separation of $W$. (b) Square lattice with finite-sized electrodes represented as black squares and voltage nodes are represented by red points. The square lattice is a mapping of the NWN in panel (a), and has $N_x = 13$ and $N_y = 7$ nodes. (c) Equivalent resistance as a function of $N_x$. Circular dots are the calculated results for the disordered NWN whereas triangular dots correspond to the results of the corresponding effective square lattice.}
{fig:interElectrode_square_nwn}

The number of parallel paths in the effective medium square lattice can be calculated using a variant of the geometric probability method used in the famous ``Buffon's Needle'' problem\cite{buffon}. Consider an input electrode with $N_y$ wire intersections, each intersection opens the possibility of a parallel path between the electrodes. In this approximation, we will take the number of electrode intersections as the number of parallel paths between electrodes. Consider a wire of length $L$, if the centre point of the wire is a distance $x<\frac{L}{2}$ from an electrode, the wire will intersect the electrode if the angle $\theta$ is in the range
\begin{equation}
0 \leq \theta \leq \cos^{-1}\left(\frac{2x}{L}\right)
\end{equation}
\noindent where $\theta$ is the angle the wire makes with the horizontal. A wire at a distance $x$ intersects the electrode with a probability $\frac{2}{\pi} \theta$. In order to obtain a probability that a wire intersects a vertical electrode axis once its center is $x \leq \frac{L}{2}$ from the electrode, we perform an integration over $\theta$ as
\begin{equation}
\frac{2}{\pi} \int_0^1 d\theta~ \cos^{-1}(\theta) = \frac{2}{\pi}
\label{eq: electrode_prob}
\end{equation}
We now consider how many wires lie in the range that they could potentially intersect the electrode. If wires are distributed homogeneously with a density of $n_w$ and over a vertical width range of W, the relevant area is $HL/2$ which contains $HLn_w/2$ wires. Combining this with the probability of electrode intersection derived in equation \ref{eq: electrode_prob}, the expected total number of intersections ($N_y$) can be written as
\begin{equation}
N_y = \frac{L H n_w}{\pi}
\label{eq:ny_theory}
\end{equation}

Figure \ref{fig:ny_theory_params} compares equation \ref{eq:ny_theory} with computer simulations in which $N_y$ was counted for systems with various wire densities (panel (a)), and lengths (panel (b)). In both cases, the analytical expression (blue dashed line) shows excellent agreement with simulations but note that equation \ref{eq:ny_theory} overestimates $N_y$ in each case, particularly for sufficiently high wire densities and lengths. These discrepancies are due to boundary effects; at the top and bottom parts of the NWN, the constraint on wire positions increases making electrode intersections in these areas less likely. At low wire lengths and densities the boundary effects do not have as much an impact $N_y$, hence the agreement between equation \ref{eq:ny_theory} and simulation is improved.

\fig{1}
{Images/Chapter4/Ny_dependence_new.pdf}
{\textbf{Plot:} Dependence of $N_y$ on various underlying parameters.}
{(a) Dependence of $N_y$ on wire densities in a NWN ensemble of size $20~ \times 20~ \mu m$ determined using equation \ref{eq:ny_theory} (blue dashed line) and computational simulations (green data points). Here the wire length was fixed at 7 $\mu m$. (b) Dependence of $N_y$ on the wire length in a NWN of size $20~ \times 20~ \mu m$ for equation \ref{eq:ny_theory} (blue dashed line) and simulations (red data points). Here the wire density was fixed at 0.4 nanowires/$\mu m^2$. The average of 20 randomly generated NWNs was used in the simulations in both plots (data points).}
{fig:ny_theory_params}

The nodal separation between electrodes is more difficult to approximate. A useful interpretation is to view a NWN as a small-world network\cite{watts1999_2} on short length scales and a regular network for larger length scales. A Watts-Strogatz (WS) network is an example of a small-world network\cite{watts1998}. A WS network is created by taking a regular lattice network where each node has $z$ nearest neighbours. A percentage $p$ of links are removed and are then used to connect random pairs of nodes anywhere else in the network. We assume that NWNs of size $L\times L$ behave as small-world networks, where $L$ is the typical length of a nanowire. The rationale here is that a current-carrying segment can act as a pathway for current flow and allow current to move a large distance at a time whereas a junction resistor does not facilitate large distance movement, current moves from one wire to another. Therefore the wire segments act as the random long range connections in a WS model but only over distances less than the length of a wire. The optimal path between two nodes in network is one that minimises the total weight. In this thesis, the weight of a network represents the resistance values and as discussed in the previous section; they follow the distribution $f(g)$ in equation \ref{eq: long_nwn_conductance_dist}. Braunstein et al\cite{braunstein2003} showed that when weak disorder is introduced to the weight values of links, the length of the optimal path in terms of the nodal separation ($q_{opt}$) that minimises the total weight of the path connecting two nodes scales as
\begin{equation}
q_{opt} \propto \frac{1}{p z^2} \log(N p z) 
\label{eq:small_world_scale}
\end{equation} 
here N is the number of nodes in the network. 

Equation \ref{eq:small_world_scale} can be used to estimate the length of the optimal path between electrodes in a NWN. In the MNR model, the number of nodes in an area $L\times L$ in terms of wire density is $2 n_j L^2$ or $2 \omega L^4 n_w^2$ using the relationship between wire length and density outlined in chapter 2\cite{ocallaco2016}. Each node is connected to one junction resistor, a wire segment and either another wire segment or a dead end making the degree of each node $z = 3$. $p$ is the percentage of current carrying intra-wire segments in the network as they can connect two nodes that have a large separation. Therefore $p = P_{cc} = \frac{2 N_j - N_w}{3N_j + N_w}$ from equation \ref{eq: resistor_percentage}. Subbing this into equation \ref{eq:small_world_scale}, $q_{opt}$ scales as
\begin{equation}
q_{opt} \propto \frac{1}{P_{cc}} \log (6 \omega L^4 n_w^2 P_{cc})
\label{eq:nwn_scaling}
\end{equation}

Consider a network of size $W \times H, ~ W>>L$, $L$ being the length of a nanowire, as in Figure \ref{fig:interElectrode_square_nwn}(a) and one of its nodes labeled A that lies on the electrode of the NWN. The optimal path between node A and node B that are separated by a distance $L$ is $q_{opt}$ as defined above. Similarly the distance between node B and another node C that are again separated by a distance $L$ is $q_{opt}$ and so two electrodes separated by a distance $W$ is $\frac{W}{L} q_{opt}$. Using equation \ref{eq:nwn_scaling}, the number of resistors in the shortest path connecting the two electrodes can be written as\cite{ocallaco2016}
\begin{equation}
N_x =\frac{W}{L} \frac{\kappa}{P_{cc}} \log (6 \omega L^4 n_w^2 P_{cc})
\label{eq:nwn_nx_theory}
\end{equation}
where $\kappa = 1.1$ is a constant of proportionality. In Figure \ref{fig:nx_theory_params}(a), the dependence of $N_x$ on the wire density is shown for NWNs of wire lengths $7~\mu m$ and a NWN size of $20~ \times 20~\mu m$. The results of the simulations are represented by the data points and equation \ref{eq:nwn_nx_theory} by the blue curve. The theoretical curve gives a reasonable approximation to the nodal separation between electrodes, however it does underestimate the path length at low densities. In Figure \ref{fig:nx_theory_params}(b), the nodal separation for given wire lengths in a network is presented. Here, simulated NWNs were of size $30~ \times 30~\mu m$ and the wire density was set to $0.4$ nanowires/$\mu m^2$. The results of the simulations are represented as green data points. Equation \ref{eq:nwn_nx_theory} is plot as the blue curve and estimates $N_x$ quite well. 
 
\fig{1}
{Images/Chapter4/nx_new.pdf}
{\textbf{Plot:} $N_x$ versus electrode separation and wire density.}
{(a) The nodal distance between electrodes $N_x$ is plot versus wire density for a sample of size $20~  \times 20~\mu m$ and wire lengths of $7~\mu m$. 20 simulations of random NWNs for a given density are performed for each data point and are represented by red data points. The blue curve is equation \ref{eq:nwn_nx_theory}. (b) $N_x$ versus wire length $L$ for a sample of size $30~\times 30~\mu m$ and a wire density of $0.5$ nanowires/$\mu m^2$. The blue curve represents equation \ref{eq:nwn_nx_theory}. In both plots, the error bars refer to the $95\%$ confidence intervals. }
{fig:nx_theory_params}

Combining the many separate parts derived in this chapter, the formula to describe the inter-electrode resistance of a NWN by means of an effective square lattice of electrode height $H$, electrode separation $W$, and effective conductance $g_m$ is calculated using the following equations
\begin{center}
\begin{align}
R_e = &R_m \frac{N_x}{N_y} = \frac{1}{g_m} \frac{N_x}{N_y} \nonumber\\
N_y = &\frac{L H n_w}{\pi}\nonumber \\
N_x = &\frac{W}{L} \frac{\kappa}{P_i} \log (6 \omega L^4 n_w^2 P_i) \nonumber \\
P_{cc} = &\frac{2 \omega L^2 n_w^2 - n_w}{3 \omega L^2 n_w^2 + n_w} \nonumber \\
\tilde{l}_s = &\frac{L n_w}{2 \omega L^2 n_w^2 + n_w}\nonumber \\
g_{cc} = &\frac{A_c}{\rho \tilde{l}_s} \nonumber \\
g_m = &\frac{g_{cc}(\omega L^2 n_w -3) - g_j (1 + \omega L^2 n_w)}{2+6\omega L^2 n_w^2} +\frac{1}{2+6\omega L^2 n_w} \times \nonumber \\
&\times \Big( 12 g_{cc} g_j(\omega L^2 n_w -1)(1+3\omega L^2 n_w) (3 g_{cc} +g_j +\omega(g_j-g_{cc})L^2 n_w)^2 \Big)^{\frac{1}{2}} 
\label{eq:combined_emt}
\end{align}
\end{center}
Here the parameters needed to calculate the sheet resistance are: the wire density $n_w$, wire length $L$, NWN device height $H$, electrode separation $W$, the junction conductance $g_j$, the wire resistivity $\rho$, and cross sectional area $A_c$. While at face value the expressions in equation \ref{eq:combined_emt} seems to have many complex constituents, they all are calculated from the fundamental parameters of the NWN. These expressions allow for the approximation of several measurable quantities in a NWN all without the need of additional simulations or image processing techniques.
%==================================
\section{Application of the Effective Square Lattice}
\label{Sec: EMT Application}
\begin{comment} 
% discussion of length and wire density
In Figure \ref{fig:wire_length_dens_theory}(a) the effective square lattice summarised in equation \ref{eq:combined_emt} is plot against average values of sheet resistance for various wire densities. Data points are the average conductance with 95\% confidence intervals of Monte Carlo simulations of NWNs of size $20~\mu m^2 \times 20~\mu m^2$, wires of length $7 ~ \mu m$ and in a network of size with other parameters corresponding to those characteristic of Ag/PVP NWNs used throughout this thesis\cite{rocha2015}. The blue curve is a visualisation of equation \ref{eq:combined_emt} and agrees closely with simulated data, lying within the confidence interval for each data point. The inset graph presents equation \ref{eq:combined_emt} in blue plot in a log-log plot with a curve $\propto n_w^{-1.7}$ meant as a guide to the eye. Two scaling regions can be identified in Figure \ref{fig:wire_length_dens_theory}, namely $n_w < 0.25$ and $n_w > 0.25$. In the low-density range, the sheet resistance has a varying scaling behaviour but converges to a power law $\propto n_w^{ -1.7}$ in the latter region.

\fig{1}
{Images/Chapter4/geometric_parameters.pdf}
{\textbf{Plot:} EMT applied to wire density and length scaling}
{ (a) The dependence of sheet resistance on wire densities. Data points represent the average sheet resistance for 20 simulations for each given wire density performed in a NWN of size $20 \mu m \times 20 \mu m$ with wires of length $6.7 \mu m$ and resistance values are those characteristic for Ag PVP wires. The 95\% confidence intervals are also plot. The theoretical sheet resistance calculated using Equation \ref{eq:combined_emt} is represented by the blue curve. The inset Log-Log figure presents the theoretical curve with alongside a power law $\propto n_w^{-1.7}$ which converge at higher wire densities. (b) The dependence of sheet resistance on the wire length. Data points represent the average sheet resistance for 20 simulations with wire densities of 0.4 $\mu m^-2$, other parameters are the same as those in (a). The theoretical values obtained by Equation \ref{eq:combined_emt} are shown as the red curve. The inset curve shows the theoretical curve in a Log-Log plot and shows a clear power law dependence between $R_s$ and wire length. MAKE INSET}
{fig:wire_length_dens_theory}

Both the wire density and wire lengths have a drastic impact on the connectivity profile of the NWN and jointly determine the important parameters $N_x$, $N_y$, and the effective resistance $R_m$. However the resistive parameters of the NWN, junction resistance, and wire resistivity and cross sectional areas, only affect the calculation of $R_m$ and their impact is shown in Figure \ref{fig:resistivity_rj}. Figure \ref{fig:resistivity_rj}(a) presents the effect of changing wire resistivity on the sheet resistance for both Monte Carlo simulations of NWNs and for the expression outlined in equation \ref{eq:combined_emt}. The approximation agrees well with the Monte Carlo in this case. Figure \ref{fig:resistivity_rj}(b) is the dependence on the junction resistance for Monte Carlo simulations. The accuracy of the approximation is not ideal in this case, clearly having a different trend to the Monte Carlo results. Recall from Chapter 3 that the dependence of the sheet resistance on the resistivity and junction resistance is linear for both parameters, in Figure \ref{fig:resistivity_rj}(a)+(b) both relationships are slightly curved for $R_j$ and $\rho$. As mentioned in the derivation of the effective medium theory, it is most accurate when the current flow distribution is homogeneous across the NWN. It may be that the increasing junction resistance results in a more localised current flow making the effective medium theory less applicable. If current flow does become inhomogeneous then the approximation of $N_y$ parallel paths also becomes less applicable, or certainly the method used to calculate $N_y$ does, which would effect the approximation of the sheet resistance. Thus the effective square lattice is most accurate for networks whose junction resistances and wire segment resistance are comparable to one another, and will become less accurate if junction dominant or intra-wire dominant resistive networks. 
\end{comment}

In this section, the effective square lattice will be used to estimate the sheet resistance of a network using known nanowire properties, and is compared with Monte Carlo simulations. Equations \ref{eq:combined_emt} are used to calculate the sheet resistance of an effective ordered lattice that incorporates the same characteristics of a random nanowire network, and while they are relatively complicated they can be easily coded in an interpreted programming language platform which can perform calculations instantaneously. This method allows for a quick examination on the necessary nanowire properties required for a desired sheet resistance. Furthermore, network parameters as discussed in chapter 3 such as the ultimate conductivity of a network or the network optimisation coefficient can be estimated very quickly with equations \ref{eq:combined_emt}. In Figure \ref{fig:wire_length_dens_theory}, the sheet resistance of an effective square lattice calculated with equations \ref{eq:combined_emt} and is plot against the wire density of a NWN. This is compared with the average results of an ensemble of simulated NWNs. Green data points are the average conductance of 20 simulations with 95\% confidence intervals of Monte Carlo simulations of NWNs of size $20 \times 20~\mu m$, wires of length $7 ~ \mu m$ with other parameters corresponding to those characteristic of Ag/PVP nanowires used throughout this thesis\cite{rocha2015}. The blue curve is a visualisation of equation \ref{eq:combined_emt} and agrees closely with the simulated data, lying within the confidence interval for each data point. The inset graph presents equation \ref{eq:combined_emt} in blue with horizontal axis transformed as $x = n_w - (n_w)_c$ alongside a red dashed curve proportional to $x^{-1.44}$. Recall from the discussion of percolation theory in chapter 3 that simulated resistances of Ag/PVP nanowire networks scales as $x^{-1.44}$. Here the sheet resistance as calculated by equation \ref{eq:combined_emt} converges onto this scaling at an approximate density of $0.25$ nanowires/$\mu m^2$, thus agreeing with the appropriate scaling according to percolation theory\cite{li2009}. Figure \ref{fig:wire_length_dens_theory} shows that the effective square lattice mapping is particularly accurate at capturing the networks scaling with nanowire density.

\fig{1}
{Images/Chapter4/wire_dens_inset.pdf}
{\textbf{Plot:} EMT applied to wire density and length scaling.}
{Dependence of sheet resistance on wire densities. Data points represent the average sheet resistance for 20 simulations for each given wire density performed in a NWN of size $20\times 20~ \mu m$ with wires of length $7~ \mu m$. The junction resistance values and resistivities are those characteristic for Ag/PVP wires\cite{rocha2015}. The 95\% confidence intervals for each $n_w$ are also plot. The theoretical sheet resistance calculated using equation \ref{eq:combined_emt} is represented by the blue curve. The inset log-log figure presents the theoretical curve alongside a power law proportional to $(n_w - (n_w)_c)^{\beta}$, and both curves converge at higher wire densities. The value of the scaling exponent $\beta=-1.44$ was determined in chapter 3 when examining the percolative scaling of a NWN sheet resistance with respect to the wire density.}
{fig:wire_length_dens_theory}

By altering the resistance parameters such as junction resistance and wire resistivity we can analyse the dependence of equations \ref{eq:combined_emt} on these parameters. The resistive parameters of the NWN, only affect the calculation of the effective conductance $g_m$ as they do not alter the connectivity profile, and their impact is shown in Figure \ref{fig:resistivity_rj}. Here parameters were set to values characteristic of Ag/PVP nanowires that have been used throughout this chapter when not being varied; $L=7$ $\mu m$, $D = 60$ $nm$, $R_j = 11~ \Omega$, $\rho = 22.6$ $n\Omega m$. Panel (a) presents the effect of changing wire resistivity on the sheet resistance for Monte Carlo simulations as red data points and for the expression outlined in equation \ref{eq:combined_emt} as the blue curve. The approximation agrees well with the Monte Carlo in this case. Figure \ref{fig:resistivity_rj}(b) shows the dependence of the sheet resistance ($R_s$) on the junction resistance for Monte Carlo simulations as red data points and the effective square lattice as the solid blue curve. Here the approximation gives a good estimate to the sheet resistance of the networks, however it underestimates it at low values of $R_j$. In Figure \ref{fig:resistivity_rj}, equation \ref{eq:combined_emt} for $R_s$ has a curved relationship with $R_j$ and $\rho$, unlike the linear relationships found for a fixed network geometry in chapter 3. Here it is the average of an ensemble of simulations for a given $R_j$ or $\rho$ that is being compared with the effective square lattice, and not a fixed network geometry as examined in chapter 3 and so the two relationships are not directly comparable. 

\begin{comment}As mentioned in the derivation of the effective medium theory, it is most accurate when the current flow distribution is homogeneous across the NWN. It may be that the increasing junction resistance results in a more localised current flow which the effective medium theory is less applicable. If current flow does become inhomogeneous then the approximation of $N_y$ parallel paths also becomes less applicable, which would effect the approximation of the sheet resistance. Thus the effective square lattice is most accurate for networks whose junction resistances and wire segment resistance are comparable to one another, and will become less accurate if junction dominant or intra-wire dominant resistive networks. \footnote{may add this paragraph and Figure \ref{fig:resistivity_rj} to an Appendix or remove it entirely}
\end{comment}
\fig{1}
{Images/Chapter4/resistance_parameters.pdf}
{\textbf{Plot:} Comparison of EMT and resistive parameter simulations.}
{(a) Effect of changing wire resistivity on the sheet resistance. The data points correspond to the average and 95\% confidence intervals of the sheet resistance calculated for a set of 10 simulations for a given resistivity and the curve corresponds to the theoretical expression in equation \ref{eq:combined_emt}. (b) Sheet resistance versus the junction resistance; again data points are from Monte Carlo simulations for 10 samples and the curve corresponds to equation \ref{eq:combined_emt}. Both (a) and (b) use networks of size $30\times 30~ \mu m$, wire density 0.4 nanowires$/\mu m^2$, and wire length of $7~ \mu m$.}
{fig:resistivity_rj}

Here, the effective square lattice mapping will be applied to the thirty experimental samples that were discussed in chapter 3 that had their SEM images processed. Table \ref{tab: nwn_exp} in Appendix \ref{appendix: experiment nwn} lists the sheet resistance and other properties of thirty experimental Ag/PVP NWN samples that were discussed at length in chapter 3. Figure \ref{fig:emt_exp} presents the experimental sheet resistance for each sample versus their wire density as red data points. The solid blue curve is the sheet resistance calculated with the effective square lattice method having the nanowire parameters set to those typical of Ag/PVP nanowires, in particular the junction resistance was set $R_j = 11 ~ \Omega$. The effective square lattice mapping provides a reasonable estimate to the sheet resistance of the experimental samples, however it does underestimate the majority of sheet resistances. Recall the characteristic junction resistances $R_j^{MNR}$ from chapter 3, where simulations of digitised networks with a homogeneous junction resistance distribution had a sheet resistance matching the experimental measurements. The characteristic junction resistances were higher than $11 ~ \Omega$ in most cases, i.e. they were contained in the range $2.28-152 ~ \Omega$. Thus the fact that equations \ref{eq:combined_emt} underestimate the sheet resistance when $R_j= 11 ~\Omega$ is unsurprising as $R_j^{MNR}$ suggests a higher junction resistance should be used in networks with fixed junction resistances when comparing with experimental samples, where junction resistances are not fixed.

In chapter 3, a distribution of junction resistances was shown in Figure \ref{fig: junc_res_dist}, in which there exists a small population of high resistance junctions, referred to as outliers. The electroforming process that was used to minimise junction resistances for the experimental samples involves increasing current-flow through the network slowly to a point where the sheet resistance is in a stable and low resistance state\cite{bellew2015}. As shall be demonstrated in the next two chapters, this leads to the emergence of many parallel low resistance pathways between the electrodes which can lead to some isolated junctions not being electroformed as others \cite{bellew2015}. As stated previously, the effective lattice requires a relatively homogeneous resistor distribution for accuracy but it can be used here to estimate the population of outliers in samples. By choosing a representative high resistance state of $R_j^h = 200 ~ \Omega$ that is a percentage $\chi$ of junction resistors, and fixing the remaining $1-\chi$ junction resistors at $R_j = 11 ~ \Omega$, the effective resistance is calculated as the weighted average of the two. Then, by tuning $\chi$ an estimate for the number of junction outliers can be achieved. In Figure \ref{fig:emt_exp}, the green dashed line corresponds to a $\chi = 10\%$ percentage of high resistance junctions and provides a much better agreement between the effective square lattice and experimental samples.

\fig{1}
{Images/Chapter4/emt_exp_data.pdf}
{\textbf{Plot:} Comparison of EMT and wire density simulations.}
{Sheet resistance ($R_s$) versus wire density measured for thirty experimental Ag/PVP NWNs listed in Table \ref{tab: nwn_exp} are plot as red data points. The blue curve is the sheet resistance versus wire density calculated using the effective square lattice summarised in equation \ref{eq:combined_emt} with a junction resistance of $11~ \Omega$ and other parameters typical to Ag/PVP nanowires \cite{rocha2015}. The green dashed curve includes the effect of outlier junction resistances and corresponds to an effective square lattice with 10\% of the junctions at $200~ \Omega$ and the remaining junctions at $11~ \Omega$. The black dotted line is the ultimate conductivity of a NWN for a given wire density as calculated by the effective square lattice where all junctions have a perfect resistance $R_j = 0~ \Omega$.}
{fig:emt_exp}

Further to approximating the sheet resistance, the effective square lattice can be used to quickly determine the optimization-capacity coefficient\cite{rocha2015} $\gamma$ which was defined in chapter 3 as
\begin{equation}
\gamma = 1 - \frac{R_0}{R_s^{EXP}}
\end{equation}
where $R_s^{EXP}$ is the experimental sheet resistance and $R_0$ is the contribution to the sheet resistance from the inner-wire resistances. Recall that values of $\gamma$ close to 1 represent networks whose conductivities can be considerably improved since their sheet resistances are far from the optimal value $R_0$. When $\gamma \rightarrow 0$, the network is close to its optimum conductivity and is unlikely that it can be further optimized. In chapter 3, $R_0$ was calculated numerically for each sample using the MNR model with $R_j \rightarrow 0$ and using the digitisation method to capture the connectivity profile of a NWN. The effective square lattice can be used to easily determine the sheet resistance when $R_j = 0$ and is shown by the black dotted line in Figure \ref{fig:emt_exp}. Using these theoretical values for $R_0$, $\gamma$ is calculated for each experimental sample and is plot against $R_j^{MNR}$ in Figure \ref{fig:emt_gamma_exp} as black data points. The green triangular data points are the numerical results obtained with the digitised network geometries that were plot in Figure \ref{fig: gamma_plots}. The theoretical approach tends to overestimate the optimization-capacity coefficient but it does reproduce the trend seen in numerical simulations quite well.

\fig{1}
{Images/Chapter4/exp_gamma_rj.pdf}
{\textbf{Plot:} Effective square lattice applied to the calculation of the Optimization-capacity coefficient.}
{Optimization-capacity coefficient obtained with numerical simulations for the thirty experimental samples listed in Table \ref{tab: nwn_exp} are plot as green triangular points. The black data points correspond to $\gamma$ where the ultimate conductivity $R_0$ was calculated using the effective medium square lattice (EMSC).} 
{fig:emt_gamma_exp}

It is in the application of the effective square lattice mapping for experimental results that its usefulness is properly illustrated. In chapter 3, the different node-voltage mappings, JDA and MNR, were combined with digitised SEM images of NWNs in order to estimate quantities such as the characteristic junction resistance and the optimization capacity coefficient. The effective square lattice is an instantaneous method to estimate different properties of a NWN that only requires fundamental network parameters and is a welcome tool for understanding the resistive properties of NWNs.

%==============================================
\section{Chapter Summary}
\label{Sec: EMTGF Conclusion}
In summary, here a method that establishes the correspondence between the sheet resistance of a heavily disordered NWN with that of an ordered network was outlined. To do so, the current-flow between junctions in a NWN was shown to scale with their separation logarithmically in the same manner as a regular square lattice in section \ref{Sec: Inter node}. Expressions for the relative percentages of types of resistors in a NWN were derived, those being inter-wire junctions, current-carrying wire segments, and dead-ends in section \ref{Sec: NWN EMT}. These expressions can be used to determine a quantitative critical wire density at which a percolating path between electrodes is impossible, which is much less than the one suggested by percolation theory. In the same section, the expressions for relative resistor percentages were used to create an effective medium, one which maps the resistive properties of a random NWN onto a regular square lattice. This mapping was used to show that the equivalent resistance and the nodal separation between pairs of junction intersections in a random NWN scaled logarithmically.

In section \ref{Sec: Inter Electrode}, the sheet resistance of a finite square lattice where two opposite sides are bounded by electrodes was shown to be caused by several identical paths of resistors connected in series. To apply this behaviour and the effective square lattice mapping to a NWN, expressions to approximate the number of parallel paths and their lengths were derived and gave reasonable estimates when compared with simulations. After combining the effective square lattice with the expressions for its size a closed form approximation for the sheet resistance of a NWN in terms of its underlying geometrical and electrical properties was obtained and is summarised at the end of section \ref{Sec: Inter Electrode}. 

The effective lattice was shown to approximate the sheet resistance well for changing fundamental parameters in the NWN, and this is depicted in section \ref{Sec: EMT Application}. The scaling between sheet resistance and wire density as calculated by the effective lattice agreed closely with simulations, converging on a power law with an exponent that was theoretically calculated in chapter 3. The effect of junction resistance and wire resistivity on the sheet resistance was estimated with the effective lattice and gave similar results to simulations. 

The true advantage of the effective lattice method is the ease of estimating the sheet resistance of a NWN while varying different parameters. To demonstrate this the effective lattice was applied to thirty experimental samples and used to gauge the percentage of high resistance junctions that were present in the samples. The effective lattice was also used to estimate the ultimate conductivity of a network which was then used to calculate the optimization coefficient instantaneously. The effective medium lattice approximation is best applied to networks where wire segments and junctions have similar resistances and do not have a dynamic response to current-flow. In the next chapter, experimental evidance for a dynamic evolution of junction conductance is presented. A model that captures this dynamic junction response is introduced, and the response of a network as a whole is examined.
%\bibliographystyle{ieeetr}  %%for ordered citations
%\bibliography{bibliography}
