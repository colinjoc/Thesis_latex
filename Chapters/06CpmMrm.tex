\chapter{Comparison of a Capacitive and Memristive Junction Activation Process}
In the previous chapter, the memristive activation of a NWN undergoing electrical stressing was introduced. This junction activation is a current driven process that requires increasing current levels through the NWN, causing the conductance of each junction to change in an analogue manner up until their final high conductance state. It was also shown that the PL+C model has activation patterns that are highly dependent on the properties of the junction; for supra-linear junction scaling the emergence of highly localised current flows through the networks takes place in a `winner-takes-all' (WTA) manner, while for linear and sub-linear exponents, current-flow deviates from the WTA behaviour by distributing the current throughout the network. The PL+C is not the only contemporary model that describes the dynamical activation of nanowire networks however. In recent publications the inter-wire junctions have been treated as a capacitor which breaks down when the potential across the junction reaches some critical value, causing an electrical connection between the nanowires\cite{nirmalraj2012,fairfield2014,fairfield2016}. The capacitive model (CPM) for junctions has been used to identify the activation voltage of the network, i.e the voltage required to begin current flow through the network. At the point of network activation, a shorting path between two electrodes is fully formed, facilitating current flow between sources. From here on, the PL+C model shall be referred to as the memristor model (MRM).

\fig{1}
{Images/Chapter6/fig1_sketch.png}
{\textbf{Sketch:} A sketch of MRM and CPM junctions in a NWN along with PVC images of NWN displaying a capacitive and memristive response.}
{(Top panels) Circuit sketches representing a NWN being described by a (a) capacitive model (CPM) and a (b) memristive model (MRM). Each lumped circuit element is assigned to model the electrical characteristics of the interwire junctions in their respective formation (capacitors) and adaptive conducting (memristors) modes. Horizontal green lines represent metallic electrodes. (Bottom panels) PVC SEM images of Ag NWN samples subjected to distinct electrical characterizations. In (c), the image was taken by holding the source voltage at 2 Volts and setting a leakage current of few hundreds of pA. The network dimensions are 200 x 200 $\mu$m and the white scale bar corresponds to 20 $\mu$m. In (d), the image was taken from a full I-V sweep with a limiting current compliance of 500 nA. The network dimensions are 100 x 100 $\mu$m and the white scale bar corresponds to 2 $\mu$m. Darker wires are grounded to the electrodes meaning that their junctions were optimized in response to the given excitation. Almost the whole network is featured in the capacitive/formation regime whereas a single WTA path is contrasted in the memristive/conducting regime\cite{ocallaco2018}. More details on this experiment can be found in our publication\cite{ocallaco2018}.}
{fig: intro-fig}

The expected response of the network to a capacitive junction breakdown activation model is hinted at in Passive Voltage Contrast (PVC) images performed at low leakage current levels. Figure \ref{fig: intro-fig} (c) presents a PVC image of a nanowire network exposed to very low leakage current (hundreds of $pA$). In this image the dark wires represent those that have the strongest electrical connection with the electrodes. This leakage current is too low to begin the memristive evolution of the junctions which remain in their pristine insulating state, equivalent to a parallel placte capacitor. Figure \ref{fig: intro-fig}(d) is a PVC image performed at $500$ $nA$, a level at which memristive evolution has begun and here a localised electrical connection is depicted by the dark wires and is a manifestation of the WTA behaviour discussed in the previous chapter. These images suggest that networks will behave differently in the CPM and MRM. These behaviours will be compared in this chapter.

The aim of this chapter is to introduce and characterise certain traits associated with the CPM and contrast this with the dynamics of networks undergoing memristive activation. This is achieved by applying the CPM and MRM to an identical network geometry, as sketched in Figures \ref{fig: intro-fig}(a) and (b). Nanowires connected by either capacitive or memristive junctions are complementary models whose applicability depends on how the networks are interrogated. The capacitive response is dominant when the network is interrogated by extremely low currents ($\sim$ pA); in this regime, each junction is represented by a capacitor which breaks down if the voltage drop across it exceeds its characteristic threshold voltage making an electrical contact between the wires. Once this occurs, the junction becomes a memristor at a high-resistance state (HRS) and sufficiently small currents can flow through it. As more current is adiabatically sourced onto the network, the memristive state of these junctions can be continuously evolved up to their respective low-resistance state (LRS). 

This chapter will highlight the main difference between the activation of a network with binary state junctions (CPM) and analogue state junctions (MRM). The layout of this chapter is as follows. The capacitive junction model along with the computational framework to apply it to a NWN is introduced in section \ref{Sec: cap_comp_model}. The activation patterns of the CPM is illustrated in section \ref{Sec: mrm_cpm_activations} and here it is contrasted with that of the MRM by applying both to an identical network geometry. In section \ref{Sec: cpm_avalanches}, the CPM is shown to have a scale free response to network perturbation resulting in mass junction activation events that can propagate across the entire network. The fault-tolerance attribute of networks in both activation models are contrasted in section \ref{Sec: fault_tol} by examining junction activations and network performance after a junction that is central to the network performance `fails`, or is removed from the mathematical graph describing the NWN. There is a chapter summary presented in section \ref{Sec: cpm_conclusion}. 


%Our findings highlight the main differences between a breakdown-based switching process (involving binary capacitive transitions) and an gradual switch (involving analogue memristive components) in complex NWN systems. NWNs made of slow-switching elements exhibit a continuous spectrum of conductance states bounded only by the junction cutoffs  $\Gamma_{\textrm{off}} = 1/R_{\textrm{off}}$ and $\Gamma_{\textrm{on}} = 1/R_{\textrm{on}}$. This mechanism is shown to be highly selective, funnelling most of the sourced current through a single path (WTA state) of NWs bridging the electrodes. NWNs composed of fast-switching elements are not as selective however, with a much larger number of junctions playing an active role in forming a pathway between electrodes. We show that NWNs containing fast-switching elements evolve in a discrete fashion with the system alternating between stages of idleness and activity. While active, one can characterize the network dynamics by computing cascade events (or avalanches) comprised of clusters of junctions being activated simultaneously and their respective time durations. On the other hand, slow-switching junctions undergo a whole spectrum of conductance levels as the network is gradually excited by the current source. In this way, a binary cascade-like characterization cannot be conducted without the consideration of a (arbitrary) threshold parameter that classifies the occurrence (or not) of an avalanche event.
%Networks of slow-switching elements do not form a critical structure and so no such cascade events are observed. Furthermore, we show that the slow-switching dynamics are fault-tolerant in response to perturbations, i.e. the network transport response is robust against junction failure. Only a minute increase in sheet conductance occurs at the moment a WTA path activation is slightly perturbed in the network. The fast-switching dynamics however are susceptible to larger-scale junction failures, requiring up to 55\% additional switching events to form a continuously active pathway between electrodes.   

%===============================================================================

\section{Capacitive Junction Model}
\label{Sec: cap_comp_model}
The capacitive model is relevant to negligible current flows through the network such that the capacitive response of junctions dominates the network properties. In this scheme, the nanowires are treated as equipotential wire segments and their connections as binary capacitors. This modeling scheme is similar to the JDA approach introduced in chapter 3 in that the junctions define the connectivity profile of the network. This assumption was taken as charge can easily move through the conductive nanowire but not across the insulating barrier in a junction. Depending on the voltage drop across the capacitive junction, it can be either non-activated ($\ket{0}$) or activated ($\ket{1}$). The capacitance state of a junction can flip from $\ket{0}\rightarrow \ket{1}$ if the voltage drop across it is larger than its associated breakdown voltage ($V_b$), hence a given junction connecting a pair of wires $(n,m)$ can be activated if $|V_\textit{n} - V_\textit{m}| \geq V_b$ where $V_\textit{n}$ ($V_\textit{m}$) is the potential at wire $n$ ($m$). The capacitor activation is characterized by a modification in the capacitance value of the junction as $C_{\textit{nm}}^0 \rightarrow C_{\textit{nm}}^1$ where $C_{\textit{nm}}^0$ is an estimated quantity determined uniquely by the characteristics of the wires and $C_{\textit{nm}}^1\rightarrow 0$ meaning that the junction has lost its capacitive properties and charge will start to flow through it. The values of $C_{\textit{nm}}^0$ are estimated by considering interwire junctions as parallel-plate capacitors with $C_{\textit{nm}}^0 = C^0 = \epsilon_r \epsilon_0 A/d$ $\forall$ $(n,m)$ pairs for the sake of simplicity. In the equation, $\epsilon_r$ is the relative permittivity of the dielectric, $\epsilon_0$ is the permittivity of the air, $A$ is the plate area, and $d$ is the plate separation. For our PVP coated Ag nanowires, we used $\epsilon_r\approx 2.5$, $d\approx 8$ nm and the area of the plates can be estimated from the nanowire diameters which range $D\sim 60-80$ nm. Assuming an ideal square area projected from two superimposed soft-body wires, $A=D^2$ resulting in $C^0 \approx 18$ attoFarads (aF). The capacitance of wire sections is not considered by the CPM due to it being negligible for metallic core nanowires. If the CPM was applied to non-metallic wires the CPM should be extended to account for the coupling capacitance of the wires themselves.

CPM simulation \cite{fairfield2016} begins by placing the whole capacitor network in contact with electrodes that source and drain a certain amount of charge $Q$, representing the charge that builds up due to the applied bias voltage. The applied charge is incremented from an initial value $Q_i$ up to a pre-defined maximum value of $Q_{\textrm{max}}$ in steps of $\Delta Q$. At $Q_i$, all junctions are set at $\ket{0}$-state, and at each charge increment the electric potential of each wire is calculated and the potential difference across each junction is checked against the breakdown voltage. A capacitance matrix $\hat{M}_c$ is built taking into account the network connectivity and the potential on each wire is obtained by solving the system of equations $\hat{M}_c \hat{V} = \hat{Q}$ self-consistently. This means that charge on the electrodes is only incremented once all $\ket{0}\rightarrow \ket{1}$ transition activity on the network ceases. A work-flow schematic of the recursive CPM is shown in Figure \ref{fig: cpm_workflow}.

\fig{0.6}
{Images/Chapter6/cpm_work_flow.pdf}
{\textbf{Schematic:} A work-flow diagram of the capacitive junction model.}
{A workflow diagram of the capacitive junction model. Refer to the main text for fetails on the algorithm.}
{fig: cpm_workflow}

%At low current levels corresponding to the OFF-threshold in MRM, one can expect to find a capacitive response from the individual NW junctions coupled with some leakage current since their dielectric coating are not expected to be an ideal insulator; a small DC current can always leak through the dielectric material. For example in the passive voltage contrast image in Figure \ref{fig: intro-fig}(c) the leakage current was of the order of hundreds of pA ($10^{-7}A$).  To account for this dual response, CPM is modified to incorporate leakage current in capacitive networks by considering a parallel RC circuit as a proxy for low current flow in NWNs. A potential difference that is placed across both elements then links the charge accumulated on the NWN with a leakage current through the resistor. The size of the leakage resistor is chosen in order to give leakage currents of the order of hundreds of pA.
%The resisitor placed in parallel can be chosen in order to give a leakage c To simulate leakage currents of the order of pA, the resistance of this dummy resistor connected in parallel to the capacitive network ($R_d$), in this case we used $R_d\sim 10^{10}\,\Omega$.

The MRM follows the algorithm that was laid out in Chapter 5. For the sake of consistency, CPM and MRM were employed on the same NWN skeleton. By using an identical network geometry for both MRM and CPM simulations, the spatial fluctuations can be removed allowing a more direct comparison between the activation dynamics of both models. Figure \ref{fig: sample_nwn}(a) is a SEM image of a NWN sample made with Ag/PVP core-shell nanowires. This NWN has a wire density of 0.47 nanowires/$\mu$m$^2$ and the average length of the wires is approximately $7\, \mu$m. After digitally capturing the geometry of this network using the method introduced in chapter 3, we estimate that this network contains a total of 963 junctions. Figure \ref{fig: sample_nwn}(b) shows a stick representation of (a) which was built from the resultant graph\cite{rocha2015}. 

\fig{1}
{Images/Chapter6/sample.png}
{\textbf{Sketch:} An SEM image of a network and alongside its digitised counterpart.}
{(a) SEM image of an Ag NWN with a wire density of 0.47 nanowires/$\mu$m$^2$ and average wire length of $7~ \mu$m. Electrodes are located at either sides of the network and the white scale bar at the bottom represents 10 $\mu$m. (b) Stick representation of the Ag NWN sample taken from (a). Black sticks represent the Ag nanowires whereas the vertical thick green lines represent the electrodes\cite{ocallaco2018}.}
{fig: sample_nwn}

The CPM model is meant to capture the dynamics of the network at extremely low current levels while the MRM is applicable at higher levels. That being said, the two models are mutually exclusive in this thesis, they do not interact and there is no consideration of a dual memcapacitive and memristive response. There are reports of a memcapacitive and memristive response existing on nanoscale junctions\cite{hartmann2017,maier2016,wakrim2016,liu2006} but as the models are applied to current regimes orders of magnitude in difference, they were studied in isolation in this work. The dual memcapacitive/memristive properties of a nanowire junction is a potentially fruitfull area worth investigating and will be a subject of future work.
%========================================================================================================================
\section{Path Formation in Capacitive and Memristive Models}
\label{Sec: mrm_cpm_activations}
Recall from chapter 5 that the hallmark of supra-linear junction scaling was the emergence of winner-takes-all paths between the electrodes. The network geometry shown in Figure \ref{fig: sample_nwn} was set to evolve in accordance to the MRM outlined in chapter 5 from which $\Gamma_{\textrm{nt}} \times I$ curves were obtained. Figure \ref{fig: conductance_activated}(a) presents the evolution of the network conductance with junctions scaling according to equation \ref{eq: junction_pl} with $A_j = 0.05$ and $\alpha_j = 1.1$. The four scaling regimes identified in chapter 5 are labeled in this plot, the initial OFF state, the transient growth (TG) where the network identifies the winner-takes-all path and begins its power-law (PL) scaling. As discussed in chapter 5, the network conductance scales as $\Gamma_{nt} = A_{nt} I^{\alpha_{nt}}$ in a self-similar manner to the junctions, i.e. $\alpha_{nt} \approx \alpha_j$. Finally the network enters the post-power-law (PPL) regime where the winner-takes-all path is fully developed and additional paths are fully activated as the current flowing through the network continues to increase. 

\fig{0.6}
{Images/Chapter6/fig3_lines.png}
{\textbf{Plot:} Wire activation during NWN memristance increase for supra-linear junction exponent.} 
{(a) Simulated conductance versus current obtained for the image processed Ag NWN shown in Figure \ref{fig: sample_nwn}. The curve was calculated with the MRM. All four distinct transport regimes discussed on the main text is depicted on panel and highlighted in different colours: (OFF) OFF-threshold, (TG) transient growth, (PL) power law, and (PPL) post-power-law. Currents are expressed in units of current (u.c.). The junction characteristics are set at $\alpha_j=1.1$ and $A_j$ = 0.05. The blue circle marks the point in the curve in which the junctions comprising the WTA paths are fully optimised at $I = 1.77$ u.c. and  $\Gamma_{\textrm{nt}} = 0.013$ mS. This point marks the disruption of the PL conducting regime. (b-c) NWN skeleton in which nanowires connected by junctions at the LRS are highlighted in black and in light grey otherwise. The NWN snapshot depicted in (c) was taken at the PPL stage at $I = 30$ u.c.. The OFF conductance region is characterised by no junction resistance change and so this is the region that the CPM is applied to. Once the junctions begin to change, the network is best described by the MRM\cite{ocallaco2018}.}
{fig: conductance_activated}

A visualisation of the activated wires at the end of the PL regime is shown in Figure \ref{fig: conductance_activated}(b). An activated wire is one who has a junction driven to the quantum of conductance and is highlighted as a black thick wire compared with the light gray thin wires that have no activated junctions associated with them. This WTA path contains 7 junctions evolved to the LRS meaning that just 0.72\% of the junctions handle most of the current-flow workload in the PL regime. As more current is sourced onto the electrodes, other conducting paths are enabled in a discrete fashion. The device gradually acquires a two-dimensional character due to the formation of parallel conductive paths. About 80 supra-linear junctions reach their optimum conductive state at $I=30$ u.c. allowing the network to distribute the input current through multiple conducting paths. This is roughly 8.3\% of the junctions taking part in the conduction process. A visualisation of the activated wires at $I = 30$ u.c. is depicted in Figure \ref{fig: conductance_activated}(c).

As stated in the previous section, the CPM applies to low current levels that are not strong enough to begin the memrisitive evolution of the junctions. This is the OFF regime of a network which is depicted in Figure \ref{fig: conductance_activated}(a) and is labeled as "capacitive" at the top of the plot. At low current levels, corresponding to the OFF regime, one can expect to find a capacitive response from the individual NW junctions coupled with some leakage current since their dielectric coating are not expected to be an ideal insulator; a small DC current can always leak through the dielectric material. For example in the passive voltage contrast image in Figure \ref{fig: intro-fig}(c) the leakage current was of the order of hundreds of pA ($10^{-7}A$)\cite{ocallaco2018}.  To account for this dual response, CPM is modified to incorporate leakage current in capacitive networks by considering a parallel RC circuit as a proxy for low current flow in NWNs. A potential difference that is placed across both elements then links the charge accumulated on the NWN with a leakage current through the resistor. The size of the leakage resistor is chosen in order to give leakage currents of the order of hundreds of pA.

Figure \ref{fig: cap_evolution}(a) shows the gradual breakdown of a capacitive network by visualising the leakage current flow required to cause an increasing charge build-up across the capacitor that is in a parallel circuit with the leakage resistor of $10^{10} ~\Omega$. One can identify a sudden increase in the required leakage current flow at 6.22 aC. A visualisation of activated wires in the NWN at this point is presented in Figure \ref{fig: cap_evolution}(b) where black wires represent those with an activated junction thus giving the wire an electrical connection with either of the electrodes. Junctions that are in contact or are near the electrodes activate easily as the potential difference builds quickest in these areas. Figure \ref{fig: cap_evolution}(c) shows the activated wires at the point when a continuous electrical path between the two electrodes has formed. The current levels through the resistor at this point is $1.3 \times 10^{-7}$ A. 

\fig{1}
{Images/Chapter6/cap_current_new.png}
{\textbf{Plot:} Wire activation in a NWN during capacitance breakdown.}
{(a) Leakage current through the parallel RC circuit as a function of the charge accumulation of the capacitive NWN. Steep jumps in current levels are clear at certain charge values and correspond to sudden activations of capacitive junctions. (b) Visualisation of the network at the first set of junction activations at 6.22 aC and leakage current of $4.2\times 10^{-8}$ A. Wires with an activated junction are in black and inactivated wires are in light grey. Figure (c) presents the activated wires when an electrical path between the electrodes is formed at $1.3\times 10^{-7}$ A and 11.78 aC. (d) Activated wires at a relatively high leakage current level at $5.7\times 10^{-7}$ A and 30 aC. Almost all junctions in the network underwent breakdown and the system is now memristive at the HRS\cite{ocallaco2018}.}
{fig: cap_evolution}

A striking difference between the CPM and MRM can be seen here; the number of junctions that are activated before path formation in CPM is much greater than in path formation in MRM. In the CPM, there are 61 junctions activated at path formation, i.e. 6.33\% of junctions compared with 0.72\% of junctions in the WTA path captured in MRM. Not only are activated junctions less concentrated in the CPM at path formation, the regions of activations are slightly different. CPM seems to favour the lower half of the network for activations while the WTA emerges in the centre of the network in the MRM. Figure \ref{fig: cap_evolution}(d) is a visualisation of the network at a late stage of activation. A large swathe of the network has been activated at this stage, much like the MRM PPL regime depicted in Figure \ref{fig: conductance_activated}(c). Note the sudden jumps in the required leakage-current flow associated with clusters of breakdown events that are crucial for the development of the memristive properties of the NWN during its adiabatic electrical stress. These jumps correspond to the sudden activation of wires in the network causing the effective capacitance of the network to drop suddenly. The current level through the resistor during capacitive activation is in the order of $10^{-7}$ A which compares favourably with current levels of hundreds of pA measured in the PVC image shown in Figure \ref{fig: intro-fig}(c) and yet well below the current levels required for junction evolution in the memristive regime. As with the memristive response of a NWN, the junctions that are activated and the order in which they do so are determined by the network connectivity. CPM cannot be applied to NWNs where percolation has not occurred as a continuous line of junctions between electrodes are required to induce voltage differences across NWN junctions.

The dynamics of linear and sub-linear are different to the supra-linear and are presented in Figure \ref{fig: conductance_other_activated} for completeness. Figure \ref{fig: conductance_other_activated}(a) is the conductance curve for the linear exponent simulaiton on the network geometry presented in Figure \ref{fig: sample_nwn}. The same initial low resistance path that emerged in the supra-linear case is activated along with additional junction activations along a second low resistance path connecting the electrodes at the bottom of the network in Figure \ref{fig: conductance_other_activated}(b). Figure \ref{fig: conductance_other_activated}(c) represents the activated junctions at 30 u.c. in the linear exponent case and shows much less activated junctions than in the supra-linear simulation shown in Figure \ref{fig: conductance_activated}(c). The sub-linear exponent simulations result in the smoothest conductance curve out of the three and is shown in Figure \ref{fig: conductance_other_activated}(d). The low resistance path that emerges in the supra-linear and linear case is evident in the sub-linear regime at path formation in Figure \ref{fig: conductance_other_activated}(e). The activated wires for the supra-linear exponent at 30 u.c. are shown in Figure \ref{fig: conductance_activated}(f).

\fig{0.75}
{Images/Chapter6/activated_wires_sub-linear.pdf}
{\textbf{Plot:} Wire activation during memristance increase in a NWN with sub-linear exponent.}
{(a) Magnified $\Gamma_{nt} \times I$ curve for the network depicted in Figure \ref{fig: sample_nwn} with a linear scaling exponent $\alpha_j = 1$. (b) The activated wires at the moment where the electrodes are bridged by a path of fully activated nanowire junctions. Note that this path lies in the centre of the network, the WTA path that emerged in the supra-linear simulation in Figure \ref{fig: conductance_activated} and a second path has begun to emerge but is not fully activated at this point. (c) Activated wires at I = 30 u.c. occupy a large swath of the network, slightly less wires are activated here than in the supra-linear simulation Figure \ref{fig: conductance_activated}(c). (d) Magnified $\Gamma_{nt} \times I$ curve for the sub-linear simulation $\alpha_j = 0.9$. The activated wires at path formation are visualised in (d), a large amount of activations are required for a fully activated path to emerge. The activated wires at $I = 30~ u.c.$ are shown in (e), again there are less activated wires than in the linear case.  }
{fig: conductance_other_activated}

Qualitatively, the regions that are activated in the CPM are also activated in the MRM, however the overlap between the two is not perfect. This non-perfect overlap is essentially a manifestation of the different activation processes, binary in the case of CPM and analogue for the MRM. The difference in activation patterns for path formation is a key contrast between the two models that have been applied to the exact same network geometry, and has been seen in all other networks that both MRM and CPM have been applied to. For a more quantitative comparison between the models, the amount of activated junctions ($\Phi$) in all network diagrams appearing in Figure \ref{fig: conductance_activated} \& \ref{fig: conductance_other_activated} and at path activation in the CPM, Figure \ref{fig: cap_evolution}(b), can be found in Figure \ref{fig: barchart}. Note that the net difference $\Delta \Phi = \Phi(\textrm{PPL}) - \Phi(\textrm{PL})$ increases with respect to $\alpha_j$, meaning that the higher exponent systems are more efficient at creating isolated low resistance paths. But more importantly, this result captures the essence of the experimental observations presented in Figure \ref{fig: intro-fig}(c-d); it contrasts the highly selective activation pattern of memristive (supra-linear) NWNs with the more distributed activations obtained when the capacitive properties of these materials are probed. The number of activations needed for path formation in the CPM is much greater predicted by the MRM, pointing to the network wide participation of the CPM in path formation. 

\fig{1}
{Images/Chapter6/barplot.pdf}
{\textbf{Plot:} Activated junctions in the CPM and MRM in different exponent regimes.}
{Number of activated junctions ($\Phi$) predicted by the capacitive (CPM) and the memristive (MRM) descriptions. Note that an activated junction in the CPM picture corresponds to a capacitor having its state flipped as $\ket{0}\rightarrow \ket{1}$ whereas in the MRM picture it corresponds to a memristor reaching its most optimized conductive state at the quantum of conductance. 61 capacitors were activated in order to create a shorting path between the electrodes. The activated junctions in the MRM is determined by those that reach their ultimate conductivity state at the moment of path formation (red) and at $I = 30 ~u.c.$ (blue) for each of the distinct exponents $\alpha_j$.}
{fig: barchart}
%==============================================================================================================================================================================
\section{Scale-Free Dynamics in Capacitive Activations}
\label{Sec: cpm_avalanches}
The sudden and large amount of junction activations, also referred to as avalanches, that give rise to the steps in leakage current in Figure \ref{fig: cap_evolution}(a) offers much insight into the scale-free response of NWNs according to CPM. Of particular interest is the distribution in avalanche sizes and their respective relaxation times recorded during the CPM evolution, which is shown in Figure \ref{fig: activation_hist}. The size of an avalanche ($s$) is defined as the number of junctions that break down at a given input charge $Q$. When at least one junction breaks down, the network self-organizes by redistributing its built-up charge throughout its remaining capacitive elements which can trigger subsequent avalanche events at the same input charge. The amount of iteration steps the network takes to relax its avalanche activity up to the point where $s = 0$ is defined as the avalanche lifetime or relaxation time ($\tau$). Figure \ref{fig: activation_hist}(a) and (b) are the normalised avalanche ($f_s$) and lifetime ($f_{\tau}$) distributions taken for an ensemble containing 3000 randomly generated NWN samples, each with a fixed wire density of 0.4 nanowires/$\mu$m$^2$ and lengths of 7 $\mu$m. Three difference network sizes are simulated to investigate if finite sized effects have an impact on the scaling of $f_s$ and $f_{\tau}$: 55 $\times$ 55 $\mu$m (blue diamonds), 60 $\times$ 60 $\mu$m (green squares), and 70 $\times$ 70 $\mu$m (orange triangles). One can observe that both distributions have a power-law trend which is indicative of scale-free critical behaviour, where a small perturbation can cause changes across the entire network, and is found in many complex models such as the sandpile, game of life, and cellular automata systems\cite{bak1988}. The $f_s$ and $f_{\tau}$ power-laws agree for the three system sizes, but at large times and avalanche sizes the data becomes noisey due to the finite size of the simulated networks. Yet, we can say that nanowire meshes operating in the capacitive mode exhibit a collective integrated response to electrical stimuli that is independent of the device size, i.e. the emerging collective dynamics of capacitive NWN systems is scale-invariant at least within certain length scales. MRM dynamics do not give rise to scale-free network wide perturbations as current perturbations propagate through the network immediately and critical states do not occur. 

\fig{1}
{Images/Chapter6/fig_avalanches_new.png}
{\textbf{Plot:} Scale-free response of NWNs to leakage current.}
{(a) Avalanche (s) and its respective (b) lifetime ($\tau$) frequency distributions in log-log scale taken for a random NWN ensemble containing over 3000 network samples of fixed wire density of 0.4 nanowires/$\mu$m$^2$ and distinct sizes of 55 $\times$ 55, 60 $\times$ 60, and 70 $\times$ 70 $\mu$m. Note that for this result to acquire statistical significance, it needs to be taken for a large ensemble of random NWN samples rather than applying CPM onto the solely image-processed NWN sample of Figure \ref{fig: sample_nwn}. The dashed lines are power law fittings that give exponents of $\beta_s = -1.25$ for the avalanche distribution sizes and $\beta_\tau = -1.42$ for the lifetime distribution. Finite size effects play an important role in cutting off the power law trend specially in the lifetime results. (c) Time traces of current response to 10.5 V DC bias measured in an Ag NWN sample of dimensions 1 $\times$ 1 mm. The network experiences this DC voltage for 20 hours in total but the plot only depicts the first three hours of measurement. (d) Fourier transform (in log-log scale) of the time traces of DC current response shown on panel (c). The power-law fit (red dashed line) gives a $1/f^\beta$ scaling with an exponent $\beta = 1.01$\cite{ocallaco2018}.}
{fig: activation_hist}

In addition to the avalanche characterization provided by the computational model, experimental evidence of the collective dynamics of NWNs operating at minimal leakage currents was found, similar to works of Avizienis et al \cite{avizienis2012} and Demis et al \cite{demis2016}. The experiment consists of measuring time traces of leakage current in a NWN sample experiencing a DC bias voltage for a large period of time. By performing a Fourier transform on the measured fluctuations in current, one can unveil complex emergent behaviours related to the activation process of the network and its recurrent connectivity structure. An Ag/PVP NWN of dimensions 1 $\times$ 1 mm was connected to a 10.5 V bias for 20 hours in total, recording the current throughout. Only the first three hours of current data is required to analyse the leakage-current response of the sample because, after three hours of measurement, sufficiently high currents levels were recorded indicating that the network had surpassed leakage conduction. These results are shown in Figure \ref{fig: activation_hist}(c-d). The presence of a power-law trend in the power spectrum points to a network-wide activation that is scale-free with a $1/f^\beta$ noise scaling with $\beta = 1.01$. As argued by Avizienis et al. \cite{avizienis2012}, such persistent current fluctuations at DC bias indicate the capacity of the network in avoiding the formation of a single dominant high-conductivity pathway between electrodes. This view agrees with the picture captured by our CPM (with a leakage term) of a scale-free clustering activation process in NWNs operating at a sufficiently low-current domain.

%Figure \ref{fig: avalanche_diff_density}(a) presents frequency distribution of the Avalanche size and (b) the time scales of avalanches for networks of wire densities 0.5 (black circles), 0.4 (red triangles), and 0.35 (blue squares) NWs/$\mu$m$^2$ in a NWN of fixed size 55 $\times$ 55 $\mu m^2$. $f_s$ and $\tau$ both follow same trend lines identified in Figures \ref{fig: activation_hist}(a) and (b). Here the changing wire and junction wire density does not seem to have a dramatic effect on the scale-free dynamics on the CPM. \todo{more discussion, talk to  claudia about this}

The CPM bears similarity to the circuit-breaker model developed by Chae et al\cite{chae2008} in which the resistance of elements in a lattice switch ON and OFF instantaneously with an applied voltage crossing a certain threshold. Unlike the CPM however the change in resistance was reversible, able to switch between high and low resistance states depending on its current state and the associated critical voltage. They too reported avalanche behaviour but did not report the power-law analysis such as that presented in Figures \ref{fig: activation_hist}(a) and (b). This suggests that the scale free avalanche behaviour is a result of the binary nature of junctions and will be a focus of future work as it has important implications to the neuromorphic computing capabilities of NWNs.
%%%%%%%%%%%%%%%%%%%%%%%%%%%%%%%%%%%%%%%%
\section{Fault Tolerance}
\label{Sec: fault_tol}
As so far demonstrated, disordered NWNs can exhibit scale-free capacitive activation or self-similar selective memristive dynamics depending on which current range the network is being probed. In particular, such memristive random networks are very attractive for probing collective features that are typical of biological neural systems such as adaptability, parallel processing, and fault-tolerance capabilities. Contrary to regular patterned devices - such as crossbar arrays \cite{prezioso2015,kim2012} - where each unit has a singular role, computation in random memristive networks relies on the non-deterministic action of their nonlinear elements distributed in a highly disordered manner. The disordered and dynamical natures of these networks make them ideal candidates to probe novel fault-tolerant computing paradigms. In other words, the massively parallel processing power characteristic of disordered interconnects combined with the adaptability of their building-blocks enables self-organization, reconfiguration, and self-healing to mitigate device shortcomings \cite{snider2007}. To illustrate such robustness to variability in random memristive NWNs, the role played by defects on their conduction and capacitive response is presented in this section.

A defect is made on a network composed of supra-linear junctions exhibiting WTA conduction by the removal of a junction from this key path before any current is applied to the network and junction evolution begins. This is a striking  perturbation to consider since in principle it can destroy the current flow through the most important network path. MRM simulations were carried out to monitor the network conductance as a function of current for the defective system and compared with the original $\Gamma_{\textrm{nt}}\times I$ curve shown in Figure \ref{fig: conductance_activated}(a). Figure \ref{fig: deleted_junction}(b) is a visualisation of the WTA path in the unperturbed network, identical to that shown in Figure \ref{fig: conductance_activated}(b). Figure \ref{fig: deleted_junction}(c) depicts the new WTA path that is formed in the perturbed network with the destroyed junction represented by the red star. The conductance evolution for both original and defective NWN is almost identical at least until the first stages of the PPL regime as shown in Figure \ref{fig: deleted_junction}(a). The self-healing properties embedded in the dynamics of memristive NWNs are clear in this example; the disruption of paths forces the junctions to re-adapt and this causes a redistribution of current across the network frame. The system then reconfigures into another least-resistance path that does not aversely impact its overall conductance using hence just a little extra power to stress this second WTA path.

\fig{0.75}
{Images/Chapter6/fig6_rough.png}
{\textbf{Plot:}  Fault-tolerance in a memristive network perturbed from start of simulations.}
{(a) $\Gamma_{\textrm{nt}} \times I$ curves obtained for the original (black dashed line) and the defective network (red line). The junction characteristics in these simulations are $\alpha_j=1.1$ and $A_j=0.05$. The curves only differ at the PPL regime. (b) Network diagram depicting wires in the WTA (black sticks) at $I = 1.77$ u.c. obtained using MRM in the original NWN from Figure \ref{fig: sample_nwn}. (c) A junction in this path was deleted and it is highlighted by a red star symbol. The network self-organizes the current transmission to another WTA path located at its bottom part. This path contains the same amount of junctions as in the original network, i.e. 7 junctions. Wires carrying residual or no current at all appear in light grey. Vertical green lines represent the electrodes that source current onto the network\cite{ocallaco2018}.}
{fig: deleted_junction}

A second type of junction failure was also simulated where the network evolves unperturbed until the formation of the WTA path and at a point in the first plateau the junction `fails` and is removed from the Kirchhoff matrix. When the junction fails, the magnitude of the applied current will not alter but the current flow through the network will dramatically reorder itself. Associated with this re-ordering, two junction responses to decreased current-flow are examined. In the first, the conductance of each junction will be allowed to either decay to a value given by $\Gamma_{j} = A_j I_j^{\alpha_j}$ representing a memristive junction that responds quickly to decreased current-flow. In the second set of simulations, junctions will not be allowed to decay to a lower conductance, representing a system with very slow or no decay of the memristive state. Figure \ref{fig: middle_deleted_junction}(a) presents the $\Gamma_{nt} \times I$ curves for the system with memristive junctions that can decay. The black dashed line depicts the conductance curve of the original NWN, i.e. with no defect. The red line is the conductance curve of the defective NWN. The dramatic spike downwards in conductance seen at $I = 2$ u.c. corresponds to the point at which the defective junction failed. Immediately after the spike the conductance recovers to a value just below that before the junction failure. The conductance oscillates around a steady state until $ I = 2.7$ u.c. where there is a spike in conductance. The rise and subsequent fall in conductance corresponds to the gradual reduction in current flow through the junctions in the original WTA, reducing their conductance. A visualization of the NWN at $ I = 3$ u.c presents the activated junctions and the wires involved in the new WTA path. None of the junctions in the old WTA path remain in their fully optimised state. 

Figure \ref{fig: middle_deleted_junction}(c) presents the conductance curve of the faulty network where no decay in conductance occurs in the original WTA junctions apart from the failed junction. Again an immediate spike downwards is seen after junction failure at I = 2 u.c and current flow is redistributed through the NWN. Incredibly, the conductance of the NWN actually increases beyond its conductance prior to junction failure. This may due be to the development of new junctions joining the original WTA and a second WTA path emerging. The conductance then increases at a steady rate afterwards towards the second plateau. Interestingly in the visualisation of the activated junctions in at I = 3 u.c. there are no additional junctions activated meaning that a new WTA has not fully developed yet. This means that the the majority of current is still being funneled through the remnants of the original WTA path. At the point of the failed junction it then flows through the undeveloped network. Both junction failure simulations show that there is an abrupt redistribution of current-flow through a NWN where a junction in the WTA fails soon after its formation. The abrupt redistribution of current-flow when a junction fails under a current load is a clear indication of the potential fault-tolerance of random NWNs. The conduction levels return to near the unperturbed system levels and their subsequent evolution is much in line with the pristine networks. 

It should be noted that here the MRM is meant to capture the gradual increase in conductance levels associated with an adiabatic increase in sourced current on the network. The sudden junction failures presented in Figure \ref{fig: middle_deleted_junction} cause a sudden redistribution of current flow that may not be properly captured by the MRM. Future investigation of the fault-tolerance of memrsitive networks will require the implementation of a modeling schemes that account for the materials response to large current-flow changes at an atomistic level\cite{hansen2015,menzel2015,menzel2017}.
%To fully understand the process behind this observation a current mapping was made at I = ... INSERT IMAGE!! \todo{current map}. 

\fig{0.75}
{Images/Chapter6/middle_del_fault.pdf}
{\textbf{Plot:} Fault-tolerance in a memristive network perturbed during simulations.}
{(a) $\Gamma_{\textrm{nt}} \times I$ curves obtained for the original (black dashed line) and the defective network (red line) where the defect was introduced during the first plateau and junction conductances are allowed to decay. (b) Network diagram depicting activated wires at $I = 3$ u.c. with junction conductance decay. The defective junction is represented by a red star, and wires with activated junctions are in black. (c) $\Gamma_{\textrm{nt}} \times I$ curves obtained for the original (black dashed line) and the defective network (red line) where the defect was introduced during the first plateau and junction conductances cannot decay. (d) Network diagram depicting activated junctions at $I = 3$ u.c. with no junction conductance decay. The defective junction is represented by a red star, activated junctions are blue dots and wires with activated junctions on them are in black. When junction conductances cannot decay, no shift in WTA takes place.}
{fig: middle_deleted_junction}

In the capacitive regime, it was demonstrated that small perturbations can have a significant effect on the network dynamics as depicted in Figure \ref{fig: activation_hist}(a-b). Here the network is perturbed by deleting a key junction that is involved in forming the path between electrodes in the CPM before any charge has begun accumulating on the electrodes. In Figure \ref{fig: cpm_start_fault}(a), the unperturbed network is presented when the leakage path between electrodes has been formed for the first time and this occurs at the charge of $11.77$ aC. Figure \ref{fig: cpm_start_fault}(b) shows the activated wires at the moment of path formation for the perturbed network with one of its crucial junctions being destroyed from the start of the simulation (represented by the red star). This junction plays a pivotal role in the dynamics of path formation in the capacitive network which is evident when we compute the number of activated wires for both pristine and perturbed cases. The unperturbed network activates 61 wires and junctions whereas the defective one mobilizes 95 wires and 126 junctions, an increase of 56\% of activated wires and over 100\% for junctions with respect to the benchmark pristine system. The charge required to form the electrode-electrode path also points to the sensitivity of the network to perturbations: $13.05$ aC for the defective NWN compared with $11.77$ aC for the unperturbed one. 

\fig{1}
{Images/Chapter6/cap_fault_comparison.png}
{\textbf{Plot:} Fault-tolerance in a capacitive network.}
{ (a) Network diagram depicting activated wires (black sticks) in the original NWN described as a capacitive system via CPM. The simulation ends when a continuous path of grounded wires is formed between the electrodes. 61 wires and junctions are activated in this simulation. (b) The same activation simulation as in (a) but with a defective junction marked with a red star symbol. The defective junction does not play any role in the simulation as it is a failed junction from the start. The simulation ends when a path is formed between electrodes, requiring 126 junction activations and 95 wire activations. (c) The same activation simulation as in (a) but with a defective junction that fails immediately after path formation. 124 junction activations or 95 wire activations occur for path reformation\cite{ocallaco2018}.}
{fig: cpm_start_fault}

A second failure method was simulated which is the failure of a key junction immediately after a continuous path is formed for the first time. The same junction as that failed in \ref{fig: cpm_start_fault}(b) was taken as the failure point. The junction was chosen to fail at $Q = 11.77$ a.c, the same point at which a path is formed in the pristine NWN. The junction is then removed from the capacitance matrix but both wires remain activated and the simulation continuous until a new path is formed between the two electrodes. Here the path reforms at 12.06 u.c, less than that required for the network in \ref{fig: cpm_start_fault}(b). Again 95 wires are activated in this network but there are 124 junctions activated, two less than in the case shown in panel (b). A visualisation of the activated wires is shown in Figure \ref{fig: cpm_start_fault}(c) and looks very similar to panel (b).

The fact that both failure mechanisms provide very similar activation behaviours may be due to the fact that the CPM is a network wide activation procedure, unlike the MRM. Contrasting the fault-tolerant results captured by CPM and MRM, one can conclude that the CPM shows a greater sensitivity to network geometry and connectivity profile. Perturbations to the connectivity profile results in wildly different activation patterns in the NWN and the destruction of key junctions involved in path formation results in a large increase in the required junction activations and acquired electrode charge. The MRM however is much more robust; while the WTA path may completely re-route when a fault is encountered it does so in an efficient manner with little change in the global conductance of the network. 

%----------------------------------------------------------------------------------------------
\section{Chapter Summary}
\label{Sec: cpm_conclusion}

In this chapter, a model that describes nanowire junctions as a binary state capacitor that transitions to an activated state once a critical potential difference across the junction was introduced in section \ref{Sec: cap_comp_model}. A computational routine to apply the capacitive model to a nanowire network with an incrementally increasing applied voltage was described in section \ref{Sec: cap_comp_model}. This chapter also addresses the path formation dynamics, scale-free response to network perturbations, and the fault-tolerance of the CPM was compared with the MRM that was introduced in chapter 5.

In section \ref{Sec: mrm_cpm_activations}, path formation in the CPM was compared with that of the memristive junction model. The two successfully model the leakage capacitive and memristive responses of NWNs perturbed at distinct transport regimes: the capacitive to extremely low current levels ($\sim$ pA) and the memristive to currents in the range of $\sim$ nA-$\mu$A. By applying both models to an identical network geometry, the contrasting dynamics of both models were highlighted. The MRM was previously shown to develop highly selective current-flow paths in a winner-takes-all manner for certain junction parameters. The CPM on the other hand displayed a network wide activation pattern before a continuous path of activated junctions existed between electrodes. Not only were the activation patterns different between the two descriptions, the emergent paths between electrodes were located in entirely different areas of the network.
 
The CPM was shown to move between periods of idleness to abrupt cascades of mass activations of junctions in the network as the potential difference across the device was increased in section \ref{Sec: cpm_avalanches}. The size frequency distribution of these activation events, or avalanches of activations, were shown to follow a power-law relationship which is indicative of scale-free complex network dynamics, i.e. the effect of perturbations to the network is only limited by the size of the network. Experimental evidence for the complex dynamics of a physical nanowire network with negligible current flow was presented and corroborates the scale-free nature predicted by CPM simulations.

The fault-tolerance of the CPM and MRM models was shown through demonstrating the response of the network to a junction failure in the main inter-electrode path in section \ref{Sec: fault_tol}. In the MRM, two types of failures were simulated. First a key junction in the WTA was destroyed before any current was sourced on the network and the current level and sheet conductance required to achieve a new WTA path were obtained. It was found that only a slight drop in sheet conductance occurred at the formation of a WTA and a negligible increase in required current was observed in the perturbed network. A second failure simulation saw a key junction in the WTA path fail when the network had already achieved a WTA path, while continuing to increase the current sourced through the network. Two junction responses to the redistribution of current through the NWN were simulated, either the junction conductance state was allowed to decay to a lower conductance value or it was not allowed to decay but only increase with increasing current levels. In both simulations, the network experienced a large and sudden drop in conductivity but quickly recovered as a new WTA was formed. Where junction conductivities were able to decay, the networks conductivity recovered to a level just below the unperturbed network's. The simulation with irreversible conductive states saw that the conductance of the networks actually increases after the failure. This highlights the robustness of massively parallel memristive networks in the MRM regime. 

The sensitivity of the CPM to junction faults was also presented in section \ref{Sec: fault_tol}, with two junction failure types similar to those examined for the MRM. Again a key junction in the formation of the shorting path in the CPM was destroyed from the beginning of the simulation and the number of activation processes required to restablish an electrode-electrode connection was recorded. In this particular example, 126 junctions were activated at the formation of the shortening path compared with the 61 activations that occurred in the unperturbed network. The second junction failure simulation saw the same key junction fail once the shortening path between electrodes was formed. Here the reconstruction of the path occurred with 124 junction activations, two less than the network perturbed from the start of the simulation. The perturbed capacitive required more than double the junction activations to reform a shorting path between the electrodes showing that junction failure in the CPM has a large effect in the capacitive properties of a network compared to the response of the memristive model where failures had a minor effect on the network conductivity.


%\bibliographystyle{ieeetr}  %%for ordered citations
%\bibliography{chap6_bibliography}
%open questions:
%Do I add current scans for fault tolerence?

